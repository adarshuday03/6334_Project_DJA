\documentclass[12pt,letterpaper]{article}

% Package imports
\usepackage[utf8]{inputenc}
\usepackage[margin=1in]{geometry}
\usepackage{amsmath}
\usepackage{amssymb}
\usepackage{graphicx}
\usepackage{hyperref}
\usepackage{setspace}
\usepackage{enumitem}
\usepackage{titlesec}
\usepackage{xcolor}
\usepackage{booktabs}
\usepackage{array}
\usepackage{colortbl}
\usepackage{float}
\usepackage{caption}

% Color definitions
\definecolor{primaryblue}{RGB}{0,51,102}
\definecolor{accentred}{RGB}{153,0,0}
\definecolor{lightgray}{RGB}{240,240,240}
\definecolor{tableheader}{RGB}{0,51,102}

% Section formatting with colored titles and rules (no numbers displayed)
\titleformat{\section}
  {\normalfont\Large\bfseries\color{primaryblue}}
  {}{0em}{}
  [\vspace{-0.3ex}\textcolor{primaryblue}{\titlerule[1.5pt]}]

\titleformat{\subsection}
  {\normalfont\large\bfseries\color{accentred}}
  {}{0em}{}

\titleformat{\subsubsection}
  {\normalfont\normalsize\bfseries\color{primaryblue}}
  {}{0em}{}

% Compact spacing for all heading levels
\titlespacing*{\section}{0pt}{1.2ex plus 0.4ex minus 0.2ex}{0.25ex plus 0.1ex}
\titlespacing*{\subsection}{0pt}{1ex plus 0.3ex minus 0.1ex}{0.25ex plus 0.1ex}
\titlespacing*{\subsubsection}{0pt}{0.8ex plus 0.2ex minus 0.1ex}{0.2ex plus 0.1ex}

% Hyperlink colors
\hypersetup{
    colorlinks=true,
    linkcolor=primaryblue,
    citecolor=primaryblue,
    urlcolor=primaryblue
}

% Custom list formatting - compact professional style
\setlist[itemize]{leftmargin=*, itemsep=0.1em, parsep=0em, topsep=0.3em}
\setlist[enumerate]{leftmargin=*, itemsep=0.1em, parsep=0em, topsep=0.3em}

% Set compact paragraph spacing
\setlength{\parskip}{0.15em}
\setlength{\parindent}{0pt}

% Enable section numbering internally but hide from display
\setcounter{secnumdepth}{3}
\setcounter{tocdepth}{3}

% Remove numbers from table of contents
\makeatletter
\renewcommand{\numberline}[1]{}
\makeatother

% Custom commands
\newcommand{\RR}{\mathbb{R}}
\newcommand{\ZZ}{\mathbb{Z}}
\newcommand{\E}{\mathbb{E}}

% Title page information
\title{
  \vspace{-1.5em}
  {\color{primaryblue}\Huge\textbf{SparkFire LLC}}\\
  \vspace{0.5em}
  {\color{accentred}\Large\textit{Newsvendor Analysis for Fireworks Distribution}}\\
  \vspace{0.8em}
  \textcolor{primaryblue}{\rule{0.85\textwidth}{2.5pt}}\\
  \vspace{0.8em}
  {\color{primaryblue}\LARGE\textbf{Optimal Order Quantity Decisions}}\\
  {\color{primaryblue}\LARGE\textbf{Under Demand Uncertainty}}\\
  \vspace{0.8em}
  \textcolor{primaryblue}{\rule{0.85\textwidth}{2.5pt}}
}

\author{
  \vspace{0.5em}
  {\color{accentred}\Large\textbf{Team DJA}}\\
  \vspace{0.8em}
  \begin{tabular}{rl}
    Dominic Jose & \href{mailto:djose35@gatech.edu}{djose35@gatech.edu}\\[0.2em]
    Jatin Patel & \href{mailto:jpatel706@gatech.edu}{jpatel706@gatech.edu}\\[0.2em]
    Adarsh Uday & \href{mailto:auday8@gatech.edu}{auday8@gatech.edu}
  \end{tabular}\\
  \vspace{1em}
  \textcolor{primaryblue}{\rule{0.85\textwidth}{2.5pt}}\\
  \vspace{0.8em}
  {\large ISyE 6334 - Stochastic OR for Supply Chain Engineering}\\
  \vspace{0.2em}
  {\large Fall 2025}\\
  \vspace{0.2em}
  {\large under Prof.\ Xin Chen}\\
  \textcolor{primaryblue}{\rule{0.85\textwidth}{2.5pt}}\\
  \vspace{0.5em}\\
  \includegraphics[height=1.5cm]{images/gt.png}\\
  \vspace{0.3em}\\
  {\small Georgia Institute of Technology}\\
  {\small H. Milton Stewart School of Industrial \& Systems Engineering}
}

\date{December 4, 2025}

\begin{document}

% Title page
\maketitle
\thispagestyle{empty}
\newpage

% Table of Contents
\tableofcontents
\newpage

%==============================================================================
\section{Executive Summary}
%==============================================================================

This report analyzes optimal order quantities for SparkFire LLC's ``Seventh Heaven'' sparkler product under demand uncertainty. We examine baseline newsvendor decisions, pricing strategies, promotional incentives, quantity discounts, and two-stage ordering with demand signals. Key ambiguity resolutions are specified task-wise in the Technical Appendix.

%------------------------------------------------------------------------------
\subsection{Task 1: Conceptual Analysis of Order Quantity}
%------------------------------------------------------------------------------

Based on cost parameters ($p=\$5$, $c=\$3$, $f=0.5$, $s=\$0.50$), we predict $Q^*$ will \textbf{EQUAL} expected demand of 270 units.

\begin{itemize}
    \item \textbf{Underage Cost:} $C_u = p - c = \$2.00$ per stockout
    \item \textbf{Overage Cost:} $C_o = c(1-f) + s = \$2.00$ per unsold unit
\end{itemize}

Since $C_u = C_o$, costs are perfectly balanced. This symmetric structure places optimal inventory at the distribution median, which equals the mean (270 units) for uniform demand.

%------------------------------------------------------------------------------
\subsection{Task 2: Optimal Order Quantity Calculation}
%------------------------------------------------------------------------------

\subsubsection{Newsvendor Model}

Cost parameters:
\begin{align*}
C_u &= p - c = 5 - 3 = \$2.00 \quad \text{(underage)} \\
C_o &= c(1-f) + s = 3(0.5) + 0.5 = \$2.00 \quad \text{(overage)}
\end{align*}

Critical ratio: $\text{CR} = C_u/(C_u + C_o) = 2.00/4.00 = 0.50$

Optimal quantity for $D \sim \text{Uniform}[120, 420]$:
\[
Q^* = a + (b-a) \times \text{CR} = 120 + 300 \times 0.50 = \boxed{270 \text{ cases}}
\]

Expected profit: $\E[\Pi(Q^*)] = \$370$.

\subsubsection{Reflection: Distribution Assumption Limitation}

The uniform distribution assumes equal likelihood for all demand values, which rarely holds in practice. Alternative distributions impact results:

\begin{itemize}
    \item \textbf{Normal:} Similar $Q^*$, higher profit (reduced variance)
    \item \textbf{Lognormal:} Higher $Q^*$ (right-skew increases stockout risk)
    \item \textbf{Impact:} Distribution choice significantly affects both optimal policy and profitability
\end{itemize}

\subsection{Task 3: Refund Sensitivity}

\begin{center}
\small
\begin{tabular}{cccc}
\toprule
Refund $f$ & $C_o$ & $Q^*$ & $\E[\Pi]$\\
\midrule
0.00 & \$3.50 & 229 & \$329\\
0.25 & \$2.75 & 246 & \$346\\
0.50 & \$2.00 & 270 & \$370\\
0.75 & \$1.25 & 305 & \$405\\
1.00 & \$0.50 & 360 & \$460\\
\bottomrule
\end{tabular}
\end{center}

Higher refund rates reduce overage cost, making excess inventory less costly. This encourages larger orders and increases expected profit. Full refund eliminates downside risk entirely, boosting profit by \$131 versus no refund. Conversely, low refund rates make SparkFire more conservative due to higher financial exposure on unsold inventory.

\subsection{Task 4: Pricing Decision}

\subsubsection{Optimal Quantity at p = \$6}

With higher price $p = \$6$ (and $f=0.5$):
\begin{itemize}
    \item $C_u = p - c = 6 - 3 = \$3.00$ (increased from \$2.00)
    \item $C_o = c(1-f) + s = \$2.00$ (unchanged)
    \item Critical ratio: $3/(3+2) = 0.60$
    \item $Q^*_{p=6} = 120 + 300 \times 0.60 = \boxed{300 \text{ cases}}$
\end{itemize}

\subsubsection{Policy Comparison \& Recommendation}

\begin{center}
\small
\begin{tabular}{lccc}
\toprule
Policy & $Q^*$ & $\E[\Pi]$ & Change\\\midrule
$(p=\$5, Q^*=270)$ & 270 & \$370 & Baseline\\
$(p=\$6, Q^*=300)$ & 300 & \$610 & +\$240 (+65\%)\\\bottomrule
\end{tabular}
\end{center}

\textbf{Recommendation:} Set $p = \$6$. $\E[\Pi]$ increases 65\% with only 11\% more inventory.

\subsubsection{Reflection: Price-Demand Relationship}

Our analysis assumes demand remains at [120,420]. If \$6 pricing reduces demand, moderate elasticity ($\sim$20\% drop) still favors \$6, but high elasticity ($\sim$40\% drop) makes \$5 optimal. 
\subsection{Task 5: Monte Carlo Simulation \& Risk Analysis}

\subsubsection{Simulation Results}

Using 500 replications with discrete demand $D \in \{120, \ldots, 420\}$ and optimal $Q^* = 270$:

\begin{center}
\small
\begin{tabular}{lc}
\toprule
\textbf{Metric} & \textbf{Value} \\
\midrule
Mean Profit & \$373.00 \\
Std Deviation & \$194.45 \\
Min Profit & -\$80.00 \\
P(Loss) & 6.8\% \\
5th Percentile & -\$22.60 \\
\bottomrule
\end{tabular}
\end{center}
Python implementation, seed=6334 - results align with theoretical expectation (\$370). Multi-seed robustness testing (seeds 6334, 1234, 5678) confirms stability.

\subsubsection{Risk Mitigation Strategy}

Break-even occurs at $D = 140$, 20-units above minimum demand (6.7\% of range).

\textbf{Recommended Action:} Order $Q \in [255, 260]$ instead of $Q^* = 270$ for downside protection.

\textbf{Trade-offs:}
\begin{itemize}
    \item Profit sacrifice: \$2 to \$3 (less than 1\%)
    \item P(Loss) reduction: 6.8\% to around 4.5\% (38\% relative improvement)
    \item Worst-case improvement: \$20 to \$30 better minimum profit
\end{itemize}

\subsubsection{Local Reflection: Weather \& Regulatory Risk}

\textbf{Weather Impact (ex: Rainy October):} assumed a 20\% demand drop
\begin{itemize}
    \item Profit erosion by around 30\%
    \item Validates need for weather monitoring pre-order
\end{itemize}

\textbf{Regulatory Risk (ex: Fireworks Ban in County):} assumed a 60\% demand drop
\begin{itemize}
    \item Loss probability of about 75\%
    \item Underscores importance of regulatory tracking
\end{itemize}

\textbf{Mitigation:} Conservative ordering ($Q=255$ to 260) combined with monitoring of weather forecasts and local regulations provides essential risk controls for local market.

\subsection{Task 6: Corvette Prize Incentive}

\subsubsection{(a) Modified Profit Model}

Expected profit with prize:
\[
\E[\Pi_{\text{prize}}(Q)] = \E[\Pi_{\text{base}}(Q)] + \text{Prize} \times P(\text{win} \mid Q)
\]

Using 5\% win probability at sales $\geq 400$ units:
\[
P(\text{win} \mid Q \geq 400) = 0.05 \times P(D \geq 400) = 0.05 \times \frac{20}{300} = 0.00333
\]

Expected prize value at $Q=400$: $\$40{,}000 \times 0.00333 = \$133$.

\subsubsection{(b) Optimal Quantity Q**}
\begin{center}
\small
\begin{tabular}{cccc}
\toprule
$Q$ & Base Profit & Prize EV & Total $\E[\Pi]$ \\
\midrule
270 & \$370 & \$0 & \$370 \\
380 & \$289 & \$160 & \$449 \\
400 & \$257 & \$214 & \$471 \\
420 & \$220 & \$213 & \$433 \\
\bottomrule
\end{tabular}
\end{center}

\textbf{Optimal decision:} $Q^{**} = 400$ units maximizes total expected profit at \$471.

\subsubsection{(c) Risk-Seeking Behavior Analysis}
\begin{itemize}
    \item $Q^* = 270$ (no prize): Safe, balanced inventory
    \item $Q^{**} = 400$ (with prize): +48\% inventory (+130 units)
    \item Base profit drops \$370 $\to$ \$257 (aggressive overstocking)
    \item Prize EV compensates: Total profit +27\% (\$370 $\to$ \$471)
\end{itemize}

\textbf{Incentive Effect:} The prize encourages \textbf{risk-seeking} behavior. SparkFire stocks 130 additional units (97\% increase in leftover inventory from 38 to 131 units) to maximize prize eligibility, accepting 31\% lower base profit for potential windfall.

\subsubsection{Assumption \& Reflection}

\textbf{Prize Rule:} We use 5\% @ 400 units as the primary threshold, consistent with the problem. The 3\% @ 380 and 7\% @ 420 are treated as contextual references from other states.
\textbf{Behavioral Limitation:} Adding expected value (\$133 to \$267) understates psychological impact. The vivid prospect of winning a \$40,000 Corvette likely drives decision-makers to order \textit{beyond} rational $Q^{**}$, as small probabilities are overweighted (Kahneman-Tversky theory). Real ordering may exceed 420 units due to regret aversion and lottery appeal.

\subsection{Task 7: Quantity Discounts}

\subsubsection{(a) Modified Profit Model}
\[
c(Q) = \begin{cases}
\$3.00 & Q \in [1, 199] \\
\$2.85 & Q \in [200, 399] \\
\$2.70 & Q \geq 400
\end{cases}
\]
\[
\E[\Pi(Q)] = p \cdot \E[\text{Sales}] + c(Q) \cdot f \cdot \E[\text{Leftover}] - s \cdot \E[\text{Leftover}] - c(Q) \cdot Q - K
\]
\subsubsection{(b) Optimal Quantity with Discounts}
\begin{center}
\small
\begin{tabular}{ccccc}
\toprule
Tier & $Q$ Candidate & Unit Cost & $\E[\text{Sales}]$ & $\E[\Pi]$ \\
\midrule
1--199 & 199 & \$3.00 & 189 & \$336 \\
200--399 & 278 & \$2.85 & 237 & \$408 \\
400+ & 400 & \$2.70 & 269 & \$358 \\
\bottomrule
\end{tabular}
\end{center}

\textbf{Optimal:} $Q^*_d = 278$ units at \$2.85/unit maximizes profit at \$408.
\subsubsection{(c) Comparison to Baseline}
\begin{itemize}
    \item \textbf{Baseline Q*} 270 units at \$3.00 to \$370 profit
    \item \textbf{With discounts Q*\textsubscript{d}:} 278 units at \$2.85 to \$408 profit
    \item \textbf{Improvement:} +\$38 (+10\%), requiring only 8 additional units
\end{itemize}
\subsubsection{(d) Supply Chain Coordination}
\begin{itemize}
    \item \textbf{Supplier benefit:} Larger orders improve economies of scale
    \item \textbf{Buyer benefit:} Cost savings offset inventory risk (\$0.15/unit reduction)
    \item \textbf{Coordination:} Shared surplus prevents extreme decision like unprofitable 400+ overorder
\end{itemize}

\subsection{Task 8: Two-Stage Ordering with Demand Signal}

Pre-season cost $c_0 = \$3.00$; expedited cost $c_1 = \$3.60$.

\textit{Stage 1 (before signal):} Order $Q^*_0 = 207$ cases.

\textit{Stage 2 (after signal):}
\begin{itemize}
    \item High demand signal: $Q^*_1(\text{H}) = 101$ additional cases (total 308)
    \item Low demand signal: $Q^*_1(\text{L}) = 0$ additional cases (total 207)
\end{itemize}

Expected profit: \$366.56. Baseline (no signal): \$370.00.

\textcolor{accentred}{VOSRC = $-\$3.40$}: The signal capability destroys value because the 20\% expedited premium ($\$3.60$ vs \$3.00) outweighs the benefit of demand information. SparkFire should not adopt this two-stage approach.

%------------------------------------------------------------------------------
\subsection{Managerial Insights}
%------------------------------------------------------------------------------

\begin{enumerate}
    \item The baseline order of 270 cases balances underage and overage risks equally (critical ratio = 0.5).
    
    \item Price increases have substantial leverage: a \$1 price increase raises optimal profit by 65\%.
    
    \item The Corvette promotion effectively shifts optimal behavior from 270 to 400 cases---a 48\% increase in order quantity.
    
    \item Quantity discounts should be evaluated holistically; the deepest discount is not always optimal.
    
    \item Demand signal value depends critically on expedited cost premiums. Here, the 20\% premium makes the signal worthless.
\end{enumerate}

\vspace{2em}

%==============================================================================
\section{Open-Ended Discussion}
%==============================================================================

% Placeholder for OE responses (pick 2 from OE1--OE4)

\textit{[This section will contain responses to two selected open-ended questions from OE1--OE4. Space reserved for detailed analysis.]}

\vspace{3cm}

%==============================================================================
% Technical Appendix
%==============================================================================

% tech.tex - Technical Appendix
% This file contains only content (no preamble/document environment)
% Include in main.tex using: % tech.tex - Technical Appendix
% This file contains only content (no preamble/document environment)
% Include in main.tex using: % tech.tex - Technical Appendix
% This file contains only content (no preamble/document environment)
% Include in main.tex using: \input{tech}

%==============================================================================
\section{Technical Appendix}
%==============================================================================

\subsection*{Computational Methodology}

All analyses employ the newsvendor model framework with demand $D \sim \text{Uniform}(120, 420)$. Calculations are performed using:
\begin{itemize}
    \item \textbf{Python 3.11:} Analytical solutions, Monte Carlo simulations, and visualization (NumPy, Matplotlib)
    \item \textbf{Microsoft Excel:} Formula-based verification and sensitivity analysis
\end{itemize}

Results presented below are primarily from Python analytical solutions. Complete Excel workbook with detailed formula documentation is included in submission materials. Full CSV datasets available in \texttt{output/csv/} directory.

%==============================================================================
\subsection{Task 1: Conceptual Analysis of Order Quantity}
%==============================================================================

\subsubsection{Overage vs. Underage Trade-off Analysis}

With selling price $p = \$5$ and wholesale cost $c = \$3$:

\textbf{Cost of Underage (lost profit per stockout):}
\[
C_u = p - c = 5 - 3 = \$2.00
\]

Every unit of unmet demand costs \$2 in lost profit margin.

\textbf{Cost of Overage (net loss per unsold unit):}
\[
C_o = c(1-f) + s = 3(1-0.5) + 0.5 = \$2.00
\]

Every unsold unit costs \$2: we paid \$3, receive \$1.50 refund (50\% of cost), and pay \$0.50 shipping to return it.

\subsubsection{Ambiguity Resolution}

The overage cost $C_o$ represents the \textit{net loss} per unsold unit. While we receive a refund of $f \cdot c = \$1.50$, we incur a shipping cost of $s = \$0.50$ to return the unit. Therefore:
\begin{align*}
\text{Net loss} &= \text{Cost} - \text{Refund} + \text{Shipping} \\
&= c - fc + s = c(1-f) + s
\end{align*}

\subsubsection{Conceptual Prediction}

\textbf{Prediction:} The optimal order quantity $Q^*$ should \textbf{EQUAL} expected demand (270 units).

\textbf{Reasoning:}
\begin{itemize}
    \item $C_u = C_o = \$2.00$ creates perfectly balanced costs—understocking and overstocking are equally penalized
    \item This symmetric cost structure requires equal weighting of stockout and overage probabilities
    \item For uniform distribution, this balance occurs at the median, which equals the mean (270 units)
    \item Mathematically: we seek $P(D \geq Q) = P(D \leq Q) = 0.5$
\end{itemize}

This prediction will be verified analytically in Task 2.

%==============================================================================
\subsection{Task 2: Optimal Order Quantity Calculation}
%==============================================================================

\subsubsection{Newsvendor Model Formulation}

\textbf{Parameters:} $p = \$5$ (selling price), $c = \$3$ (unit cost), $f = 0.5$ (refund fraction), $s = \$0.50$ (shipping cost per return), $K = \$20$ (fixed ordering cost).

\textbf{Cost Structure:}
\begin{align*}
C_u &= p - c = 5 - 3 = \$2.00 \quad \text{(underage cost: lost profit per stockout)}\\
C_o &= c(1-f) + s = 3(1-0.5) + 0.5 = \$2.00 \quad \text{(overage cost: net loss per unsold unit)}
\end{align*}

\textbf{Critical Ratio:}
\[
\text{CR} = \frac{C_u}{C_u + C_o} = \frac{2.00}{2.00 + 2.00} = 0.5000
\]

\textbf{Optimal Order Quantity:}
\[
Q^* = a + (b-a) \times \text{CR} = 120 + (420-120) \times 0.5 = \boxed{270 \text{ units}}
\]

\textbf{Verification:} This confirms our conceptual prediction from Task 1. The balanced cost structure ($C_u = C_o$) results in $Q^*$ exactly at the expected demand.

\subsubsection{Profit Breakdown at $Q^* = 270$}

\begin{figure}[H]
\centering
\begin{minipage}{0.48\textwidth}
    \centering
    \small
    \setlength{\tabcolsep}{4pt}
    \begin{tabular}{lrl}
    \toprule
    \textbf{Component} & \textbf{Value} & \textbf{Calculation} \\
    \midrule
    Expected Sales & 232.50 units & $\E[\min(D, Q^*)]$ \\
    Expected Leftover & 37.50 units & $Q^* - \E[\text{Sales}]$ \\
    \midrule
    Revenue & \$1,162.50 & $232.50 \times \$5$ \\
    Salvage Value & \$56.25 & $37.50 \times \$3 \times 0.5$ \\
    Shipping Cost & $-$\$18.75 & $37.50 \times \$0.50$ \\
    Ordering Cost & $-$\$20.00 & Fixed \\
    Variable Cost & $-$\$810.00 & $270 \times \$3$ \\
    \midrule
    \textcolor{accentred}{\textbf{Expected Profit}} & \textcolor{accentred}{\textbf{\$370.00}} & Total \\
    \bottomrule
    \end{tabular}
    \captionof{table}{Profit decomposition at $Q^* = 270$ units}
\end{minipage}
\hfill
\begin{minipage}{0.45\textwidth}
    \centering
    \includegraphics[width=\textwidth]{../output/plots/q1_q2_profit_curve.png}
    \caption{Expected profit curve with $Q^*$}
\end{minipage}
\end{figure}

\subsubsection{Distribution Sensitivity Analysis}

The optimal solution depends on the assumed demand distribution. Below we compare $Q^*$ and expected profit across alternative distributions with similar central tendency:

\begin{table}[H]
\centering
\small
\begin{tabular}{lccc}
\toprule
\textbf{Distribution} & \textbf{Parameters} & \textbf{$Q^*$} & \textbf{$\E[\text{Profit}]$} \\
\midrule
Uniform & $[120, 420]$ & 270 & \$370 \\
Normal & $\mu=270, \sigma=50$ & 270 & \$385 \\
Lognormal & $\mu=270$, right-skewed & 285 & \$368 \\
Triangular & $[120, 270, 420]$ & 255 & \$372 \\
\bottomrule
\end{tabular}
\caption{Impact of distribution choice on optimal policy (same cost parameters)}
\label{tab:dist_sensitivity}
\end{table}
Normal distribution yields higher profit due to concentrated probability around the mean. Lognormal (right-skewed) shifts $Q^*$ upward to hedge against high-demand tail risk. Triangular (mode-centered) reduces optimal order slightly. Distribution choice materially affects both policy and performance.

%==============================================================================
\subsection{Task 3: Refund Sensitivity Analysis}
%==============================================================================

We evaluate how refund generosity affects optimal ordering decisions across the full spectrum from no refunds to full refunds: $f \in \{0.00, 0.25, 0.50, 0.75, 1.00\}$.

\subsubsection{Sensitivity Results}

\begin{table}[H]
\centering
\small
\setlength{\tabcolsep}{4pt}
\begin{tabular}{ccccccc}
\toprule
\textbf{Refund} & \textbf{$C_o$} & \textbf{$C_u$} & \textbf{Critical} & \textbf{$Q^*$} & \textbf{$\E[\text{Profit}]$} \\
\textbf{Rate $f$} & & & \textbf{Ratio} & \textbf{(units)} & \\
\midrule
0.00 & \$3.50 & \$2.00 & 0.3636 & 229.1 & \$329.09 \\
0.25 & \$2.75 & \$2.00 & 0.4211 & 246.3 & \$346.32 \\
0.50 & \$2.00 & \$2.00 & 0.5000 & 270.0 & \$370.00 \\
0.75 & \$1.25 & \$2.00 & 0.6154 & 304.6 & \$404.62 \\
1.00 & \$0.50 & \$2.00 & 0.8000 & 360.0 & \$460.00 \\
\bottomrule
\end{tabular}
\caption{Refund sensitivity analysis across full spectrum}
\label{tab:q3_sensitivity}
\end{table}

\textbf{Observations:}
\begin{itemize}
    \item Higher refund rates systematically reduce overage cost $C_o$, increasing critical ratio and $Q^*$
    \item Boundary cases show full range: $Q^*$ from 229 units (no refund) to 360 units (full refund)
    \item Expected profit increases monotonically from \$329 ($f=0$) to \$460 ($f=1$)—a \$131 gain
    \item Full refund ($f=1.00$) essentially eliminates overage risk ($C_o = \$0.50$ shipping only), encouraging aggressive ordering
\end{itemize}

\begin{figure}[H]
\centering
\begin{minipage}{0.48\textwidth}
    \centering
    \includegraphics[width=\textwidth]{../output/plots/q3_Qstar_vs_refund.png}
    \caption*{(a) Optimal $Q^*$ vs refund rate}
\end{minipage}
\hfill
\begin{minipage}{0.48\textwidth}
    \centering
    \includegraphics[width=\textwidth]{../output/plots/q3_profit_vs_refund.png}
    \caption*{(b) Expected profit vs refund rate}
\end{minipage}
\caption{Impact of refund policy on order quantity and profitability}
\label{fig:q3_sensitivity}
\end{figure}

%==============================================================================
\subsection{Task 4: Pricing Decision}
%==============================================================================

We compare two pricing strategies : $p = \$5$ (baseline) vs $p = \$6$ (premium pricing).

\subsubsection{Conceptual Prediction for $p = \$6$}

\textbf{At $p = \$6$:}
\begin{itemize}
    \item $C_u = p - c = 6 - 3 = \$3$ (increased from \$2)
    \item $C_o = c(1-f) + s = \$2$ (unchanged)
    \item Critical ratio: $C_u/(C_u + C_o) = 3/5 = 0.60 > 0.50$
\end{itemize}

\textbf{Prediction:} $Q^*$ should exceed 270 units. Higher underage cost shifts strategy toward more inventory to reduce stockout risk.

\subsubsection{Comparative Analysis}

\begin{table}[H]
\centering
\small
\setlength{\tabcolsep}{6pt}
\begin{tabular}{lcccccc}
\toprule
\textbf{Price} & \textbf{$C_u$} & \textbf{Critical} & \textbf{$Q^*$} & \textbf{$\E[\text{Sales}]$} & \textbf{$\E[\text{Leftover}]$} & \textbf{$\E[\text{Profit}]$} \\
& & \textbf{Ratio} & \textbf{(units)} & \textbf{(units)} & \textbf{(units)} & \\
\midrule
\$5 (baseline) & \$2.00 & 0.5000 & 270.0 & 232.50 & 37.50 & \$370.00 \\
\$6 & \$3.00 & 0.6000 & 300.0 & 246.00 & 54.00 & \$610.00 \\
\midrule
\multicolumn{6}{r}{\textbf{Profit Increase:}} & \textcolor{accentred}{+\$240.00 (+64.9\%)} \\
\bottomrule
\end{tabular}
\caption{Pricing comparison: \$5 vs \$6 (Q4)}
\label{tab:q4_pricing}
\end{table}

\textbf{Recommendation:} Set $p = \$6$. The higher price increases expected profit by 65\% while requiring only 11\% more inventory (300 vs 270 units). The increased underage cost ($C_u$ rises from \$2 to \$3) justifies stocking more units to capture higher per-unit margins.

\subsubsection{Price Elasticity Sensitivity Analysis}

\textbf{Critical Assumption:} The preceding analysis assumes demand is \textbf{price-inelastic} (demand remains Uniform$(120, 420)$ regardless of price). This is unrealistic for most products.

\textbf{Real-World Consideration:} A price increase from \$5 to \$6 (20\% hike) would likely reduce demand. Let's explore three plausible elasticity scenarios:

\begin{table}[H]
\centering
\small
\setlength{\tabcolsep}{5pt}
\begin{tabular}{llcccc}
\toprule
\textbf{Scenario} & \textbf{Demand} & \textbf{Mean} & \textbf{$Q^*$} & \textbf{$\E[\Pi]$} & \textbf{vs \$5} \\
& \textbf{Distribution} & \textbf{Demand} & & \textbf{@ \$6} & \textbf{baseline} \\
\midrule
\textit{Baseline (p=\$5)} & Uniform(120, 420) & 270 & 270 & \$370 & --- \\[0.5ex]
\midrule
A: Inelastic & Uniform(120, 420) & 270 & 300 & \$610 & \textcolor{primaryblue}{+65\%} \\
B: Moderate & Uniform(96, 336) & 216 & 240 & \$448 & \textcolor{primaryblue}{+21\%} \\
C: High elasticity & Uniform(72, 252) & 162 & 180 & \$286 & \textcolor{accentred}{$-$23\%} \\
\bottomrule
\end{tabular}
\caption{Price elasticity scenarios at $p = \$6$ (Q4 sensitivity)}
\label{tab:q4_elasticity}
\end{table}

\textbf{Scenario Details:}
\begin{itemize}
    \item \textbf{A (Inelastic):} Demand unchanged—upper bound on \$6 profit
    \item \textbf{B (Moderate):} 20\% demand reduction—mean drops to 216
    \item \textbf{C (High elasticity):} 40\% demand reduction—common for discretionary products
\end{itemize}

\textbf{Strategic Implications:}
\begin{enumerate}
    \item \textbf{Moderate elasticity} (Scenario B) still favors \$6 pricing with +21\% profit gain
    \item \textbf{High elasticity} (Scenario C) makes \$6 pricing detrimental—profit falls 23\% below \$5 baseline
    \item \textbf{Decision criterion:} Price elasticity of demand must be better than $-$2.0 for \$6 to outperform \$5
\end{enumerate}

\textbf{Recommendation:} Before implementing \$6 pricing, conduct market research or A/B testing to estimate true price elasticity. If elasticity is moderate ($|\varepsilon| < 1.0$), proceed with premium pricing. If highly elastic ($|\varepsilon| > 2.0$), maintain \$5 pricing to preserve volume.

%==============================================================================
\subsection{Task 5: Risk \& Simulation Analysis}
%==============================================================================
\subsubsection{Ambiguity Resolution - Continuous vs Discrete Demand}

For simulation realism, we use discrete uniform demand $D \in \{120, 121, \ldots, 420\}$ with equal probability (1/301 each). Analytical tasks (1--4, 6--8) use continuous approximation.

\subsubsection{Simulation Setup and Random Number Generation}

To simulate demand realizations, we use Python's \texttt{random.randint(a, b)} function, which generates discrete uniform random integers over $[a, b]$ with equal probability $1/(b-a+1)$

\textbf{Algorithm:}
\begin{enumerate}
    \item \textbf{Seed:} Set \texttt{random.seed(6334)} to ensure reproducibility and identical sequence of random numbers
    \item \textbf{Demand generation:} For each of 500 trials, generate $D_i \in \{120, 121, \ldots, 420\}$
    \item \textbf{Profit calculation:} Compute $\Pi(Q^*, D_i)$ using:
    \begin{itemize}
        \item Sales: $\min(D_i, Q^*)$
        \item Leftover: $\max(0, Q^* - D_i)$
        \item Profit: $p \cdot \text{Sales} + f \cdot c \cdot \text{Leftover} - s \cdot \text{Leftover} - K - c \cdot Q^*$
    \end{itemize}
\end{enumerate}

\textbf{Note:} Excel workbook provides iteration-level detail.

\subsubsection{Multi-Seed Robustness Verification}

We verify the simulation's robustness by running with three different seeds:

\begin{table}[H]
\centering
\small
\setlength{\tabcolsep}{4pt}
\begin{tabular}{cccccc}
\toprule
\textbf{Random} & \textbf{Mean} & \textbf{Std Dev} & \textbf{Min} & \textbf{P(Loss)} & \textbf{5th Pct} \\
\textbf{Seed} & \textbf{Profit} & & \textbf{Profit} & & \\
\midrule
6334 & \$373.00 & \$194.45 & -\$80.00 & 6.8\% & \$-22.60 \\
1234 & \$379.12 & \$191.72 & -\$80.00 & 5.4\% & \$-7.80 \\
5678 & \$387.86 & \$189.09 & -\$76.00 & 6.2\% & \$5.20 \\
\midrule
\textbf{Average} & \textbf{\$380.00} & \textbf{\$191.75} & & \textbf{6.1\%} & \\
\bottomrule
\end{tabular}
\caption{Multi-seed robustness check (Q5)}
\label{tab:q5_multiseed}
\end{table}

\textbf{Conclusion:} Mean profits cluster tightly around theoretical \$370, confirming simulation validity. The 6.1\% average loss probability indicates moderate downside risk under baseline assumptions.

\begin{figure}[H]
\centering
\includegraphics[width=0.9\textwidth]{../output/plots/q5_multiseed_comparison.png}
\caption{Profit distribution across three random seeds (Q5)}
\label{fig:q5_multiseed}
\end{figure}

\subsubsection{Excel Implementation: Single-Iteration Transparency}

While Python provides aggregate statistics across 500 trials, the Excel workbook offers iteration-level visibility for pedagogical transparency and formula verification.

\begin{figure}[H]
\centering
\begin{minipage}{0.48\textwidth}
    \centering
    \includegraphics[width=\textwidth]{images/q5_excel_first5iter.png}
    \caption*{(a) First 5 of 500 iterations}
\end{minipage}
\hfill
\begin{minipage}{0.48\textwidth}
    \centering
    \includegraphics[width=\textwidth]{images/q5_excel_sim_results.png}
    \caption*{(b) Summary statistics (500 trials)}
\end{minipage}

\vspace{0.5em}

\begin{minipage}{0.48\textwidth}
    \centering
    \includegraphics[width=\textwidth]{images/q5_expvsactualprofit.png}
    \caption*{(c) Expected vs actual profit}
\end{minipage}
\hfill
\begin{minipage}{0.48\textwidth}
    \centering
    \includegraphics[width=\textwidth]{images/q5_netprofit_500sims_excel.png}
    \caption*{(d) Net profit distribution}
\end{minipage}
\caption{Excel simulation detail: iteration-level transparency (Q5)}
\label{fig:q5_excel}
\end{figure}

Excel results confirm Python findings: mean profit \$370--\$380 range, 6--7\% loss probability, substantial profit variance (\$190+ std dev).

\subsubsection{Break-Even Analysis and Conservative Ordering Strategy}

\textbf{Critical Question:} At what demand does profit become zero?

Setting $\Pi(Q, D) = 0$ and solving for $D$:
\begin{align*}
p \cdot D + f \cdot c \cdot (Q - D) - s \cdot (Q - D) - K - c \cdot Q &= 0 \\
D \cdot (p - fc + s) &= K + Q \cdot (c - fc + s) \\
D^* &= \frac{K + Q \cdot (c - fc + s)}{p - fc + s}
\end{align*}

For $Q^* = 270$: $D^* = \frac{20 + 270(3 - 1.5 + 0.5)}{5 - 1.5 + 0.5} = \frac{560}{4} = \boxed{140}$ cases.

Profit becomes zero at $D = 140$. With minimum demand at 120, only a \textbf{20-unit buffer} exists—just 6.7\% of the 300-unit demand range. This narrow margin makes the optimal policy vulnerable to even slight demand underperformance.

\textbf{Conservative Strategy Evaluation:}

To expand the buffer zone, we evaluate order quantities below $Q^* = 270$. Lower $Q$ raises the break-even demand point, creating more cushion above the minimum. 

\begin{table}[H]
\centering
\small
\setlength{\tabcolsep}{3pt}
\begin{tabular}{ccccccc}
\toprule
\textbf{$Q$} & \textbf{Break-even $D$} & \textbf{Buffer} & \textbf{Mean Profit} & \textbf{Std Dev} & \textbf{P(Loss)} & \textbf{Min Profit} \\
\midrule
255 & 131 & 11 units (3.7\%) & \$370.42 & \$171.24 & 4.2\% & -\$50.00 \\
260 & 135 & 15 units (5.0\%) & \$371.17 & \$175.42 & 5.0\% & -\$60.00 \\
265 & 138 & 18 units (6.0\%) & \$371.91 & \$179.51 & 5.8\% & -\$70.00 \\
\rowcolor{lightgray} 270 & 140 & 20 units (6.7\%) & \textbf{\$372.64} & \$183.53 & 6.8\% & -\$80.00 \\
\bottomrule
\end{tabular}
\caption{Conservative strategy evaluation: buffer vs profit trade-off (Q5)}
\label{tab:q5_conservative}
\end{table}

\textbf{Analysis:}
\begin{itemize}
    \item Reducing $Q$ from 270 to 255 expands buffer from 6.7\% to 3.7\% of demand range
    \item Profit sacrifice minimal: \$372.64 → \$370.42 (only \$2.22 or 0.6\%)
    \item P(Loss) drops from 6.8\% to 4.2\% (38\% relative reduction)
    \item Worst-case improves by \$30 (-\$50 vs -\$80)
\end{itemize}

\textbf{Recommendation:} 
Order $Q \in [255, 260]$ to balance risk mitigation with profit preservation. The \$2 to \$3 profit sacrifice buys substantial downside protection, appropriate for risk-averse or capital-constrained operations.

\subsubsection{Demand Shock Scenarios: Weather and Regulatory Risk}

All prior analysis assumes demand remains Uniform$(120, 420)$ regardless of external conditions. This is unrealistic.
\begin{itemize}
    \item \textbf{Weather:} Heavy rain during July 4th weekend reduces outdoor celebrations
    \item \textbf{Regulatory:} Sudden fireworks ban due to drought/fire risk
\end{itemize}

If demand drops while $Q^* = 270$ is already ordered, SparkFire faces overstocking losses.

\begin{table}[H]
\centering
\small
\setlength{\tabcolsep}{4pt}
\begin{tabular}{lcccccc}
\toprule
\textbf{Scenario} & \textbf{Demand} & \textbf{Mean} & \textbf{Mean} & \textbf{Profit} & \textbf{Min} & \textbf{P(Loss)} \\
 & \textbf{Reduction} & \textbf{Demand} & \textbf{Profit} & \textbf{Change} & \textbf{Profit} & \\
\midrule
Baseline & 0\% & 270 & \$373.00 & --- & -\$80 & 6.8\% \\
Mild shock & 10\% & 243 & \$328.91 & -11.8\% & -\$128 & 12.0\% \\
Moderate (weather) & 20\% & 216 & \$259.95 & \textcolor{accentred}{-30.3\%} & -\$176 & 18.6\% \\
Severe & 40\% & 162 & \$95.82 & \textcolor{accentred}{-74.3\%} & -\$272 & 36.8\% \\
Catastrophic (ban) & 60\% & 108 & \textcolor{accentred}{-\$132.28} & \textcolor{accentred}{-135.5\%} & -\$368 & \textcolor{accentred}{75.6\%} \\
\bottomrule
\end{tabular}
\caption{Demand shock impact on profitability with fixed Q* = 270}
\label{tab:q5_shock}
\end{table}

\textbf{Critical Findings:}
\begin{itemize}
    \item 20\% demand reduction (weather) leads to 30\% profit erosion to \$216
    \item 60\% demand reduction (regulatory) has expected \textbf{loss} of \$132
    \item The profit function is \textbf{highly sensitive} to demand shocks when $Q$ is fixed
\end{itemize}

\subsubsection{Risk Mitigation Strategies}

Based on comprehensive risk analysis, we propose three mitigation strategies:

\textbf{Strategy 1: Pre-Order Intelligence \& Adaptive Ordering}
\begin{itemize}
    \item Monitor 10-day weather forecasts and regulatory developments before finalizing order
    \item If adverse conditions detected, reduce $Q$ to range [245, 260] based on risk severity
    \item Default to conservative $Q \in [255, 260]$ to expand buffer and reduce loss probability
\end{itemize}

\textbf{Strategy 2: Supplier Relationship Management}
\begin{itemize}
    \item Negotiate higher refund rate $f$ with Leisure Limited (target 75\% vs current 50\%)
    \item Consider partial ordering: 200 units initially, option for 50 to 70 additional units as needed
\end{itemize}

\textbf{Strategy 3: Diversified Sales Channels}
\begin{itemize}
    \item Establish spot market relationships for post-holiday discount sales
    \item Explore regional fireworks retailers as secondary buyers for excess inventory
\end{itemize}

%==============================================================================
\subsection{Task 6: Corvette Prize Incentive}
%==============================================================================

Leisure Limited offers a \$40,000 Corvette prize to the stand with highest statewide sales. We use Excel-based conditional probability analysis to evaluate Q* under prize incentives.

\subsubsection{Ambiguity Resolution: Prize Rule Selection}
\begin{itemize}
    \item 5\% chance if sales $\geq 400$ units (primary)
    \item 3\% chance if sales $\geq 380$ units (other states context)
    \item 7\% chance if sales $\geq 420$ units (other states context)
\end{itemize}
We use the \textbf{5\% @ 400 units} rule as the baseline analysis, treating 380 and 420 thresholds as contextual references. This aligns with the emphasis on the 400-unit threshold.

\subsubsection{(a) Modified Profit Model}

Expected profit with prize incentive:
\[
\E[\Pi_{\text{total}}(Q)] = \E[\Pi_{\text{base}}(Q)] + \E[\text{Prize} \mid Q]
\]

where base profit follows standard newsvendor model:
\[
\E[\Pi_{\text{base}}(Q)] = p \cdot \E[\min(D,Q)] + f \cdot c \cdot \E[(Q-D)^+] - s \cdot \E[(Q-D)^+] - K - c \cdot Q
\]

Prize component depends on order quantity:
\[
\E[\text{Prize} \mid Q] = \begin{cases}
0 & \text{if } Q < 400 \\
\text{Prize} \times P(\text{win}) \times P(D \geq 400) & \text{if } Q \geq 400
\end{cases}
\]

For Uniform$(120, 420)$ demand:
\[
P(D \geq 400) = \frac{420 - 400}{420 - 120} = \frac{20}{300} = 0.0667
\]

\textbf{Expected prize at $Q \geq 400$:}
\[
\E[\text{Prize}] = \$40{,}000 \times 0.05 \times 0.0667 = \$133.33
\]

\subsubsection{(b) Optimal Quantity Q** Calculation}

\textbf{Conditional Probability Approach:}

For each candidate $Q$, we partition demand into regions and calculate conditional expected profits.

\textbf{Example: $Q = 400$}

Demand regions:
\begin{itemize}
    \item Region 1: $D < 400$ with probability $P(D < 400) = 280/300 = 0.9333$
    \item Region 2: $D \geq 400$ with probability $P(D \geq 400) = 20/300 = 0.0667$
\end{itemize}

\textbf{Region 1} ($D < 400$): Expected sales = $\frac{120 + 400}{2} = 260$
\[
\E[\Pi \mid D < 400] = 5(260) + 0.5(3)(140) - 0.5(140) - 20 - 3(400) = \$250.67
\]

\textbf{Region 2} ($D \geq 400$): Sales = 400, no leftover, prize eligible
\[
\E[\Pi \mid D \geq 400] = 5(400) - 20 - 3(400) + 40{,}000(0.05) = \$2{,}780
\]

\textbf{Total expected profit:}
\[
\E[\Pi_{\text{total}}(400)] = 0.9333(\$250.67) + 0.0667(\$2{,}780) = \$419.54
\]

\textbf{Candidate Evaluation:}

\begin{table}[H]
\centering
\small
\setlength{\tabcolsep}{5pt}
\begin{tabular}{ccccccc}
\toprule
\textbf{$Q$} & \textbf{$\E[\text{Sales}]$} & \textbf{$\E[\text{Leftover}]$} & \textbf{Base} & \textbf{Prize} & \textbf{Total} & \textbf{Note} \\
 & & & \textbf{Profit} & \textbf{EV} & \textbf{$\E[\Pi]$} & \\
\midrule
270 & 232.5 & 37.5 & \$370 & \$0 & \$370 & Baseline Q* \\
380 & 267.0 & 113.0 & \$289 & \$160 & \$449 & Above threshold \\
\rowcolor{lightgray} 400 & 269.0 & 131.0 & \$257 & \$214 & \textbf{\$471} & \textbf{Optimal Q**} \\
420 & 270.0 & 150.0 & \$220 & \$213 & \$433 & Max threshold \\
\bottomrule
\end{tabular}
\caption{Profit analysis at candidate order quantities (Excel-based calculations)}
\label{tab:q6_candidates}
\end{table}

\textbf{Optimal Decision:} $Q^{**} = 400$ units maximizes total expected profit at \$471.

\textit{Note:} Complete iteration table (Q = 120 to 420) available in the Excel file, confirming optimal at boundary.

\subsubsection{(c) Risk-Seeking Behavior Analysis}

\textbf{Comparison: $Q^*$ vs $Q^{**}$}

\begin{table}[H]
\centering
\small
\begin{tabular}{lccc}
\toprule
\textbf{Metric} & \textbf{$Q^* = 270$} & \textbf{$Q^{**} = 400$} & \textbf{Change} \\
\midrule
Order Quantity & 270 & 400 & +130 (+48\%) \\
Expected Sales & 232.5 & 269.0 & +36.5 \\
Expected Leftover & 37.5 & 131.0 & +93.5 (+249\%) \\
Base Profit & \$370 & \$257 & -\$113 \\
Expected Prize & \$0 & \$214 & +\$214 \\
\midrule
\textbf{Total E[Profit]} & \textbf{\$370} & \textbf{\$471} & \textbf{+\$101 (+27\%)} \\
\bottomrule
\end{tabular}
\caption{Prize incentive impact: Q* vs Q** (Q6)}
\label{tab:q6_comparison}
\end{table}
\begin{enumerate}
    \item \textbf{Significant inventory increase:} $Q^{**}$ is 48\% higher than baseline
    \item \textbf{Base profit deteriorates:} Aggressive overstocking reduces base profit by \$113 (-31\%)
    \item \textbf{Prize compensates:} Expected prize value (\$214) offsets base profit loss
    \item \textbf{Net benefit:} Total profit increases 27\% (\$370 $\to$ \$471)
\end{enumerate}
The prize incentive induces \textbf{strong risk-seeking behavior}:
\begin{itemize}
    \item SparkFire accepts 300\% increase in expected leftover inventory
    \item Base profitability declines, but prize eligibility compensates
    \item Decision shifts from conservative (balanced $C_u=C_o$) to aggressive for high-sales threshold.
\end{itemize}
\subsubsection{Behavioral Economics: EV Model Limitations}
Our model adds expected prize value (\$133 to \$267) to base profit, treating the Corvette as a monetary equivalent. This yields $Q^{**} = 420$.

\textbf{Behavioral Reality (Kahneman-Tversky Theory):}
Real decision-makers likely \textit{over-order beyond} $Q^{**} = 420$ due to:
\begin{enumerate}
    \item \textbf{Probability weight:} Small probabilities are psychologically overweighted: 5\% \textit{feels} 20\%
    \item \textbf{Framing effect:} Win a \$40,000 Corvette is vivid and appealing, abstract \$267 EV isn't
    \item \textbf{Regret aversion:} Fear of ``almost winning'' drives extra buffer ordering
    \item \textbf{Non-linear utility:} Marginal utility of \$40k windfall far exceeds utility of \$267 EV.
\end{enumerate}

\textbf{Practical Implication:}

Lottery-style incentives exploit behavioral biases. While $Q^{**} = 420$ is actuarially optimal, actual orders may reach 450 to 500 units as managers chase the psychologically compelling prize, sacrificing expected profit for emotional appeal.

%==============================================================================
\subsection{Task 7: Quantity Discounts}
%==============================================================================

The wholesaler offers all-units quantity discounts with tiered pricing:
\[
c(Q) = \begin{cases}
\$3.00 & \text{if } Q \in [1, 199] \\
\$2.85 & \text{if } Q \in [200, 399] \\
\$2.70 & \text{if } Q \geq 400
\end{cases}
\]

\subsubsection{Tier-by-Tier Analysis}

For each cost tier, we compute the unconstrained newsvendor optimal $Q^*$, then evaluate feasibility within tier bounds.

\begin{table}[H]
\centering
\small
\setlength{\tabcolsep}{5pt}
\begin{tabular}{lccccccc}
\toprule
\textbf{Tier} & \textbf{Unit} & \textbf{$C_o$} & \textbf{$C_u$} & \textbf{Critical} & \textbf{Unconstrained} & \textbf{In} & \textbf{Candidate} \\
\textbf{Range} & \textbf{Cost} & & & \textbf{Ratio} & \textbf{$Q^*$} & \textbf{Range?} & \textbf{$Q$} \\
\midrule
1--199 & \$3.00 & \$2.00 & \$2.00 & 0.5000 & 270.0 & No & 199 \\
200--399 & \$2.85 & \$1.93 & \$2.15 & 0.5276 & 278.3 & Yes & 278 \\
400+ & \$2.70 & \$1.85 & \$2.30 & 0.5542 & 286.3 & No & 400 \\
\bottomrule
\end{tabular}
\caption{Discount tier feasibility analysis (Q7)}
\label{tab:q7_tiers}
\end{table}

\begin{itemize}
    \item \textbf{Tier 1 (\$3.00):} Unconstrained $Q^* = 270$ exceeds tier maximum (199), so evaluate boundary $Q = 199$
    \item \textbf{Tier 2 (\$2.85):} Unconstrained $Q^* = 278$ falls within [200, 399], this is a feasible interior solution
    \item \textbf{Tier 3 (\$2.70):} Unconstrained $Q^* = 286$ below tier minimum (400), so evaluate boundary $Q = 400$
\end{itemize}

\subsubsection{Candidate Profit Comparison}
\begin{table}[H]
\centering
\small
\setlength{\tabcolsep}{6pt}
\begin{tabular}{cccccc}
\toprule
\textbf{$Q$} & \textbf{Unit Cost} & \textbf{$\E[\text{Sales}]$} & \textbf{$\E[\text{Leftover}]$} & \textbf{$\E[\text{Profit}]$} & \textbf{Note} \\
\midrule
199 & \$3.00 & 189 & 10 & \$336 & Tier 1 max \\
\rowcolor{lightgray} 278 & \$2.85 & 237 & 41 & \textbf{\$408} & Tier 2 optimal \\
400 & \$2.70 & 269 & 131 & \$358 & Tier 3 min \\
\bottomrule
\end{tabular}
\caption{Candidate order quantities and expected profits (Q7, Excel-based)}
\label{tab:q7_candidates}
\end{table}

\textbf{Optimal Decision:} $Q^*_d = 278$ units at \$2.85/unit
\begin{itemize}
    \item Middle tier (\$2.85) dominates despite not having the lowest unit cost
    \item Ordering 400 units to access \$2.70 pricing forces excessive overage (131 units expected leftover)
    \item Overage cost penalty outweighs \$0.15/unit savings: total cost increases by \$50.41
\end{itemize}

\subsubsection{Comparison to Baseline}

\begin{figure}[H]
\centering
\begin{minipage}{0.45\textwidth}
    \centering
    \small
    \begin{tabular}{rccc}
    \toprule
    \textbf{Scenario} & \textbf{$Q^*$} & \textbf{Unit Cost} & \textbf{$\E[\text{Profit}]$} \\
    \midrule
    Baseline & 270 & \$3.00 & \$370 \\
    Discount Tiers & 278 & \$2.85 & \$408 \\
    \midrule
    \textbf{Improvement} & +8 units & $-$\$0.15 & \textcolor{accentred}{+\$38} \\
    \bottomrule
    \end{tabular}
    \captionof{table}{Quantity discount benefit vs baseline (Q7)}
    \label{tab:q7_comparison}
\end{minipage}
\hfill
\begin{minipage}{0.43\textwidth}
    \centering
    \includegraphics[width=\textwidth]{../output/plots/q7_profit_by_tier.png}
    \caption{Expected profit across discount tiers}
    \label{fig:q7_profit}
\end{minipage}
\end{figure}

The quantity discount structure increases expected profit by 10\% while requiring minimal additional inventory (8 units).

\textbf{Supply Chain Coordination Insight:} Quantity discounts align supplier and buyer incentives by encouraging larger orders (beneficial for supplier's economies of scale) while sharing cost savings with the buyer. The tiered structure prevents extreme ordering behavior; the marginal benefit of the deepest discount (400+ tier) is insufficient to justify the inventory risk.

%==============================================================================
% End of Technical Appendix
%==============================================================================


%==============================================================================
\section{Technical Appendix}
%==============================================================================

\subsection*{Computational Methodology}

All analyses employ the newsvendor model framework with demand $D \sim \text{Uniform}(120, 420)$. Calculations are performed using:
\begin{itemize}
    \item \textbf{Python 3.11:} Analytical solutions, Monte Carlo simulations, and visualization (NumPy, Matplotlib)
    \item \textbf{Microsoft Excel:} Formula-based verification and sensitivity analysis
\end{itemize}

Results presented below are primarily from Python analytical solutions. Complete Excel workbook with detailed formula documentation is included in submission materials. Full CSV datasets available in \texttt{output/csv/} directory.

%==============================================================================
\subsection{Task 1: Conceptual Analysis of Order Quantity}
%==============================================================================

\subsubsection{Overage vs. Underage Trade-off Analysis}

With selling price $p = \$5$ and wholesale cost $c = \$3$:

\textbf{Cost of Underage (lost profit per stockout):}
\[
C_u = p - c = 5 - 3 = \$2.00
\]

Every unit of unmet demand costs \$2 in lost profit margin.

\textbf{Cost of Overage (net loss per unsold unit):}
\[
C_o = c(1-f) + s = 3(1-0.5) + 0.5 = \$2.00
\]

Every unsold unit costs \$2: we paid \$3, receive \$1.50 refund (50\% of cost), and pay \$0.50 shipping to return it.

\subsubsection{Ambiguity Resolution}

The overage cost $C_o$ represents the \textit{net loss} per unsold unit. While we receive a refund of $f \cdot c = \$1.50$, we incur a shipping cost of $s = \$0.50$ to return the unit. Therefore:
\begin{align*}
\text{Net loss} &= \text{Cost} - \text{Refund} + \text{Shipping} \\
&= c - fc + s = c(1-f) + s
\end{align*}

\subsubsection{Conceptual Prediction}

\textbf{Prediction:} The optimal order quantity $Q^*$ should \textbf{EQUAL} expected demand (270 units).

\textbf{Reasoning:}
\begin{itemize}
    \item $C_u = C_o = \$2.00$ creates perfectly balanced costs—understocking and overstocking are equally penalized
    \item This symmetric cost structure requires equal weighting of stockout and overage probabilities
    \item For uniform distribution, this balance occurs at the median, which equals the mean (270 units)
    \item Mathematically: we seek $P(D \geq Q) = P(D \leq Q) = 0.5$
\end{itemize}

This prediction will be verified analytically in Task 2.

%==============================================================================
\subsection{Task 2: Optimal Order Quantity Calculation}
%==============================================================================

\subsubsection{Newsvendor Model Formulation}

\textbf{Parameters:} $p = \$5$ (selling price), $c = \$3$ (unit cost), $f = 0.5$ (refund fraction), $s = \$0.50$ (shipping cost per return), $K = \$20$ (fixed ordering cost).

\textbf{Cost Structure:}
\begin{align*}
C_u &= p - c = 5 - 3 = \$2.00 \quad \text{(underage cost: lost profit per stockout)}\\
C_o &= c(1-f) + s = 3(1-0.5) + 0.5 = \$2.00 \quad \text{(overage cost: net loss per unsold unit)}
\end{align*}

\textbf{Critical Ratio:}
\[
\text{CR} = \frac{C_u}{C_u + C_o} = \frac{2.00}{2.00 + 2.00} = 0.5000
\]

\textbf{Optimal Order Quantity:}
\[
Q^* = a + (b-a) \times \text{CR} = 120 + (420-120) \times 0.5 = \boxed{270 \text{ units}}
\]

\textbf{Verification:} This confirms our conceptual prediction from Task 1. The balanced cost structure ($C_u = C_o$) results in $Q^*$ exactly at the expected demand.

\subsubsection{Profit Breakdown at $Q^* = 270$}

\begin{figure}[H]
\centering
\begin{minipage}{0.48\textwidth}
    \centering
    \small
    \setlength{\tabcolsep}{4pt}
    \begin{tabular}{lrl}
    \toprule
    \textbf{Component} & \textbf{Value} & \textbf{Calculation} \\
    \midrule
    Expected Sales & 232.50 units & $\E[\min(D, Q^*)]$ \\
    Expected Leftover & 37.50 units & $Q^* - \E[\text{Sales}]$ \\
    \midrule
    Revenue & \$1,162.50 & $232.50 \times \$5$ \\
    Salvage Value & \$56.25 & $37.50 \times \$3 \times 0.5$ \\
    Shipping Cost & $-$\$18.75 & $37.50 \times \$0.50$ \\
    Ordering Cost & $-$\$20.00 & Fixed \\
    Variable Cost & $-$\$810.00 & $270 \times \$3$ \\
    \midrule
    \textcolor{accentred}{\textbf{Expected Profit}} & \textcolor{accentred}{\textbf{\$370.00}} & Total \\
    \bottomrule
    \end{tabular}
    \captionof{table}{Profit decomposition at $Q^* = 270$ units}
\end{minipage}
\hfill
\begin{minipage}{0.45\textwidth}
    \centering
    \includegraphics[width=\textwidth]{../output/plots/q1_q2_profit_curve.png}
    \caption{Expected profit curve with $Q^*$}
\end{minipage}
\end{figure}

\subsubsection{Distribution Sensitivity Analysis}

The optimal solution depends on the assumed demand distribution. Below we compare $Q^*$ and expected profit across alternative distributions with similar central tendency:

\begin{table}[H]
\centering
\small
\begin{tabular}{lccc}
\toprule
\textbf{Distribution} & \textbf{Parameters} & \textbf{$Q^*$} & \textbf{$\E[\text{Profit}]$} \\
\midrule
Uniform & $[120, 420]$ & 270 & \$370 \\
Normal & $\mu=270, \sigma=50$ & 270 & \$385 \\
Lognormal & $\mu=270$, right-skewed & 285 & \$368 \\
Triangular & $[120, 270, 420]$ & 255 & \$372 \\
\bottomrule
\end{tabular}
\caption{Impact of distribution choice on optimal policy (same cost parameters)}
\label{tab:dist_sensitivity}
\end{table}
Normal distribution yields higher profit due to concentrated probability around the mean. Lognormal (right-skewed) shifts $Q^*$ upward to hedge against high-demand tail risk. Triangular (mode-centered) reduces optimal order slightly. Distribution choice materially affects both policy and performance.

%==============================================================================
\subsection{Task 3: Refund Sensitivity Analysis}
%==============================================================================

We evaluate how refund generosity affects optimal ordering decisions across the full spectrum from no refunds to full refunds: $f \in \{0.00, 0.25, 0.50, 0.75, 1.00\}$.

\subsubsection{Sensitivity Results}

\begin{table}[H]
\centering
\small
\setlength{\tabcolsep}{4pt}
\begin{tabular}{ccccccc}
\toprule
\textbf{Refund} & \textbf{$C_o$} & \textbf{$C_u$} & \textbf{Critical} & \textbf{$Q^*$} & \textbf{$\E[\text{Profit}]$} \\
\textbf{Rate $f$} & & & \textbf{Ratio} & \textbf{(units)} & \\
\midrule
0.00 & \$3.50 & \$2.00 & 0.3636 & 229.1 & \$329.09 \\
0.25 & \$2.75 & \$2.00 & 0.4211 & 246.3 & \$346.32 \\
0.50 & \$2.00 & \$2.00 & 0.5000 & 270.0 & \$370.00 \\
0.75 & \$1.25 & \$2.00 & 0.6154 & 304.6 & \$404.62 \\
1.00 & \$0.50 & \$2.00 & 0.8000 & 360.0 & \$460.00 \\
\bottomrule
\end{tabular}
\caption{Refund sensitivity analysis across full spectrum}
\label{tab:q3_sensitivity}
\end{table}

\textbf{Observations:}
\begin{itemize}
    \item Higher refund rates systematically reduce overage cost $C_o$, increasing critical ratio and $Q^*$
    \item Boundary cases show full range: $Q^*$ from 229 units (no refund) to 360 units (full refund)
    \item Expected profit increases monotonically from \$329 ($f=0$) to \$460 ($f=1$)—a \$131 gain
    \item Full refund ($f=1.00$) essentially eliminates overage risk ($C_o = \$0.50$ shipping only), encouraging aggressive ordering
\end{itemize}

\begin{figure}[H]
\centering
\begin{minipage}{0.48\textwidth}
    \centering
    \includegraphics[width=\textwidth]{../output/plots/q3_Qstar_vs_refund.png}
    \caption*{(a) Optimal $Q^*$ vs refund rate}
\end{minipage}
\hfill
\begin{minipage}{0.48\textwidth}
    \centering
    \includegraphics[width=\textwidth]{../output/plots/q3_profit_vs_refund.png}
    \caption*{(b) Expected profit vs refund rate}
\end{minipage}
\caption{Impact of refund policy on order quantity and profitability}
\label{fig:q3_sensitivity}
\end{figure}

%==============================================================================
\subsection{Task 4: Pricing Decision}
%==============================================================================

We compare two pricing strategies : $p = \$5$ (baseline) vs $p = \$6$ (premium pricing).

\subsubsection{Conceptual Prediction for $p = \$6$}

\textbf{At $p = \$6$:}
\begin{itemize}
    \item $C_u = p - c = 6 - 3 = \$3$ (increased from \$2)
    \item $C_o = c(1-f) + s = \$2$ (unchanged)
    \item Critical ratio: $C_u/(C_u + C_o) = 3/5 = 0.60 > 0.50$
\end{itemize}

\textbf{Prediction:} $Q^*$ should exceed 270 units. Higher underage cost shifts strategy toward more inventory to reduce stockout risk.

\subsubsection{Comparative Analysis}

\begin{table}[H]
\centering
\small
\setlength{\tabcolsep}{6pt}
\begin{tabular}{lcccccc}
\toprule
\textbf{Price} & \textbf{$C_u$} & \textbf{Critical} & \textbf{$Q^*$} & \textbf{$\E[\text{Sales}]$} & \textbf{$\E[\text{Leftover}]$} & \textbf{$\E[\text{Profit}]$} \\
& & \textbf{Ratio} & \textbf{(units)} & \textbf{(units)} & \textbf{(units)} & \\
\midrule
\$5 (baseline) & \$2.00 & 0.5000 & 270.0 & 232.50 & 37.50 & \$370.00 \\
\$6 & \$3.00 & 0.6000 & 300.0 & 246.00 & 54.00 & \$610.00 \\
\midrule
\multicolumn{6}{r}{\textbf{Profit Increase:}} & \textcolor{accentred}{+\$240.00 (+64.9\%)} \\
\bottomrule
\end{tabular}
\caption{Pricing comparison: \$5 vs \$6 (Q4)}
\label{tab:q4_pricing}
\end{table}

\textbf{Recommendation:} Set $p = \$6$. The higher price increases expected profit by 65\% while requiring only 11\% more inventory (300 vs 270 units). The increased underage cost ($C_u$ rises from \$2 to \$3) justifies stocking more units to capture higher per-unit margins.

\subsubsection{Price Elasticity Sensitivity Analysis}

\textbf{Critical Assumption:} The preceding analysis assumes demand is \textbf{price-inelastic} (demand remains Uniform$(120, 420)$ regardless of price). This is unrealistic for most products.

\textbf{Real-World Consideration:} A price increase from \$5 to \$6 (20\% hike) would likely reduce demand. Let's explore three plausible elasticity scenarios:

\begin{table}[H]
\centering
\small
\setlength{\tabcolsep}{5pt}
\begin{tabular}{llcccc}
\toprule
\textbf{Scenario} & \textbf{Demand} & \textbf{Mean} & \textbf{$Q^*$} & \textbf{$\E[\Pi]$} & \textbf{vs \$5} \\
& \textbf{Distribution} & \textbf{Demand} & & \textbf{@ \$6} & \textbf{baseline} \\
\midrule
\textit{Baseline (p=\$5)} & Uniform(120, 420) & 270 & 270 & \$370 & --- \\[0.5ex]
\midrule
A: Inelastic & Uniform(120, 420) & 270 & 300 & \$610 & \textcolor{primaryblue}{+65\%} \\
B: Moderate & Uniform(96, 336) & 216 & 240 & \$448 & \textcolor{primaryblue}{+21\%} \\
C: High elasticity & Uniform(72, 252) & 162 & 180 & \$286 & \textcolor{accentred}{$-$23\%} \\
\bottomrule
\end{tabular}
\caption{Price elasticity scenarios at $p = \$6$ (Q4 sensitivity)}
\label{tab:q4_elasticity}
\end{table}

\textbf{Scenario Details:}
\begin{itemize}
    \item \textbf{A (Inelastic):} Demand unchanged—upper bound on \$6 profit
    \item \textbf{B (Moderate):} 20\% demand reduction—mean drops to 216
    \item \textbf{C (High elasticity):} 40\% demand reduction—common for discretionary products
\end{itemize}

\textbf{Strategic Implications:}
\begin{enumerate}
    \item \textbf{Moderate elasticity} (Scenario B) still favors \$6 pricing with +21\% profit gain
    \item \textbf{High elasticity} (Scenario C) makes \$6 pricing detrimental—profit falls 23\% below \$5 baseline
    \item \textbf{Decision criterion:} Price elasticity of demand must be better than $-$2.0 for \$6 to outperform \$5
\end{enumerate}

\textbf{Recommendation:} Before implementing \$6 pricing, conduct market research or A/B testing to estimate true price elasticity. If elasticity is moderate ($|\varepsilon| < 1.0$), proceed with premium pricing. If highly elastic ($|\varepsilon| > 2.0$), maintain \$5 pricing to preserve volume.

%==============================================================================
\subsection{Task 5: Risk \& Simulation Analysis}
%==============================================================================
\subsubsection{Ambiguity Resolution - Continuous vs Discrete Demand}

For simulation realism, we use discrete uniform demand $D \in \{120, 121, \ldots, 420\}$ with equal probability (1/301 each). Analytical tasks (1--4, 6--8) use continuous approximation.

\subsubsection{Simulation Setup and Random Number Generation}

To simulate demand realizations, we use Python's \texttt{random.randint(a, b)} function, which generates discrete uniform random integers over $[a, b]$ with equal probability $1/(b-a+1)$

\textbf{Algorithm:}
\begin{enumerate}
    \item \textbf{Seed:} Set \texttt{random.seed(6334)} to ensure reproducibility and identical sequence of random numbers
    \item \textbf{Demand generation:} For each of 500 trials, generate $D_i \in \{120, 121, \ldots, 420\}$
    \item \textbf{Profit calculation:} Compute $\Pi(Q^*, D_i)$ using:
    \begin{itemize}
        \item Sales: $\min(D_i, Q^*)$
        \item Leftover: $\max(0, Q^* - D_i)$
        \item Profit: $p \cdot \text{Sales} + f \cdot c \cdot \text{Leftover} - s \cdot \text{Leftover} - K - c \cdot Q^*$
    \end{itemize}
\end{enumerate}

\textbf{Note:} Excel workbook provides iteration-level detail.

\subsubsection{Multi-Seed Robustness Verification}

We verify the simulation's robustness by running with three different seeds:

\begin{table}[H]
\centering
\small
\setlength{\tabcolsep}{4pt}
\begin{tabular}{cccccc}
\toprule
\textbf{Random} & \textbf{Mean} & \textbf{Std Dev} & \textbf{Min} & \textbf{P(Loss)} & \textbf{5th Pct} \\
\textbf{Seed} & \textbf{Profit} & & \textbf{Profit} & & \\
\midrule
6334 & \$373.00 & \$194.45 & -\$80.00 & 6.8\% & \$-22.60 \\
1234 & \$379.12 & \$191.72 & -\$80.00 & 5.4\% & \$-7.80 \\
5678 & \$387.86 & \$189.09 & -\$76.00 & 6.2\% & \$5.20 \\
\midrule
\textbf{Average} & \textbf{\$380.00} & \textbf{\$191.75} & & \textbf{6.1\%} & \\
\bottomrule
\end{tabular}
\caption{Multi-seed robustness check (Q5)}
\label{tab:q5_multiseed}
\end{table}

\textbf{Conclusion:} Mean profits cluster tightly around theoretical \$370, confirming simulation validity. The 6.1\% average loss probability indicates moderate downside risk under baseline assumptions.

\begin{figure}[H]
\centering
\includegraphics[width=0.9\textwidth]{../output/plots/q5_multiseed_comparison.png}
\caption{Profit distribution across three random seeds (Q5)}
\label{fig:q5_multiseed}
\end{figure}

\subsubsection{Excel Implementation: Single-Iteration Transparency}

While Python provides aggregate statistics across 500 trials, the Excel workbook offers iteration-level visibility for pedagogical transparency and formula verification.

\begin{figure}[H]
\centering
\begin{minipage}{0.48\textwidth}
    \centering
    \includegraphics[width=\textwidth]{images/q5_excel_first5iter.png}
    \caption*{(a) First 5 of 500 iterations}
\end{minipage}
\hfill
\begin{minipage}{0.48\textwidth}
    \centering
    \includegraphics[width=\textwidth]{images/q5_excel_sim_results.png}
    \caption*{(b) Summary statistics (500 trials)}
\end{minipage}

\vspace{0.5em}

\begin{minipage}{0.48\textwidth}
    \centering
    \includegraphics[width=\textwidth]{images/q5_expvsactualprofit.png}
    \caption*{(c) Expected vs actual profit}
\end{minipage}
\hfill
\begin{minipage}{0.48\textwidth}
    \centering
    \includegraphics[width=\textwidth]{images/q5_netprofit_500sims_excel.png}
    \caption*{(d) Net profit distribution}
\end{minipage}
\caption{Excel simulation detail: iteration-level transparency (Q5)}
\label{fig:q5_excel}
\end{figure}

Excel results confirm Python findings: mean profit \$370--\$380 range, 6--7\% loss probability, substantial profit variance (\$190+ std dev).

\subsubsection{Break-Even Analysis and Conservative Ordering Strategy}

\textbf{Critical Question:} At what demand does profit become zero?

Setting $\Pi(Q, D) = 0$ and solving for $D$:
\begin{align*}
p \cdot D + f \cdot c \cdot (Q - D) - s \cdot (Q - D) - K - c \cdot Q &= 0 \\
D \cdot (p - fc + s) &= K + Q \cdot (c - fc + s) \\
D^* &= \frac{K + Q \cdot (c - fc + s)}{p - fc + s}
\end{align*}

For $Q^* = 270$: $D^* = \frac{20 + 270(3 - 1.5 + 0.5)}{5 - 1.5 + 0.5} = \frac{560}{4} = \boxed{140}$ cases.

Profit becomes zero at $D = 140$. With minimum demand at 120, only a \textbf{20-unit buffer} exists—just 6.7\% of the 300-unit demand range. This narrow margin makes the optimal policy vulnerable to even slight demand underperformance.

\textbf{Conservative Strategy Evaluation:}

To expand the buffer zone, we evaluate order quantities below $Q^* = 270$. Lower $Q$ raises the break-even demand point, creating more cushion above the minimum. 

\begin{table}[H]
\centering
\small
\setlength{\tabcolsep}{3pt}
\begin{tabular}{ccccccc}
\toprule
\textbf{$Q$} & \textbf{Break-even $D$} & \textbf{Buffer} & \textbf{Mean Profit} & \textbf{Std Dev} & \textbf{P(Loss)} & \textbf{Min Profit} \\
\midrule
255 & 131 & 11 units (3.7\%) & \$370.42 & \$171.24 & 4.2\% & -\$50.00 \\
260 & 135 & 15 units (5.0\%) & \$371.17 & \$175.42 & 5.0\% & -\$60.00 \\
265 & 138 & 18 units (6.0\%) & \$371.91 & \$179.51 & 5.8\% & -\$70.00 \\
\rowcolor{lightgray} 270 & 140 & 20 units (6.7\%) & \textbf{\$372.64} & \$183.53 & 6.8\% & -\$80.00 \\
\bottomrule
\end{tabular}
\caption{Conservative strategy evaluation: buffer vs profit trade-off (Q5)}
\label{tab:q5_conservative}
\end{table}

\textbf{Analysis:}
\begin{itemize}
    \item Reducing $Q$ from 270 to 255 expands buffer from 6.7\% to 3.7\% of demand range
    \item Profit sacrifice minimal: \$372.64 → \$370.42 (only \$2.22 or 0.6\%)
    \item P(Loss) drops from 6.8\% to 4.2\% (38\% relative reduction)
    \item Worst-case improves by \$30 (-\$50 vs -\$80)
\end{itemize}

\textbf{Recommendation:} 
Order $Q \in [255, 260]$ to balance risk mitigation with profit preservation. The \$2 to \$3 profit sacrifice buys substantial downside protection, appropriate for risk-averse or capital-constrained operations.

\subsubsection{Demand Shock Scenarios: Weather and Regulatory Risk}

All prior analysis assumes demand remains Uniform$(120, 420)$ regardless of external conditions. This is unrealistic.
\begin{itemize}
    \item \textbf{Weather:} Heavy rain during July 4th weekend reduces outdoor celebrations
    \item \textbf{Regulatory:} Sudden fireworks ban due to drought/fire risk
\end{itemize}

If demand drops while $Q^* = 270$ is already ordered, SparkFire faces overstocking losses.

\begin{table}[H]
\centering
\small
\setlength{\tabcolsep}{4pt}
\begin{tabular}{lcccccc}
\toprule
\textbf{Scenario} & \textbf{Demand} & \textbf{Mean} & \textbf{Mean} & \textbf{Profit} & \textbf{Min} & \textbf{P(Loss)} \\
 & \textbf{Reduction} & \textbf{Demand} & \textbf{Profit} & \textbf{Change} & \textbf{Profit} & \\
\midrule
Baseline & 0\% & 270 & \$373.00 & --- & -\$80 & 6.8\% \\
Mild shock & 10\% & 243 & \$328.91 & -11.8\% & -\$128 & 12.0\% \\
Moderate (weather) & 20\% & 216 & \$259.95 & \textcolor{accentred}{-30.3\%} & -\$176 & 18.6\% \\
Severe & 40\% & 162 & \$95.82 & \textcolor{accentred}{-74.3\%} & -\$272 & 36.8\% \\
Catastrophic (ban) & 60\% & 108 & \textcolor{accentred}{-\$132.28} & \textcolor{accentred}{-135.5\%} & -\$368 & \textcolor{accentred}{75.6\%} \\
\bottomrule
\end{tabular}
\caption{Demand shock impact on profitability with fixed Q* = 270}
\label{tab:q5_shock}
\end{table}

\textbf{Critical Findings:}
\begin{itemize}
    \item 20\% demand reduction (weather) leads to 30\% profit erosion to \$216
    \item 60\% demand reduction (regulatory) has expected \textbf{loss} of \$132
    \item The profit function is \textbf{highly sensitive} to demand shocks when $Q$ is fixed
\end{itemize}

\subsubsection{Risk Mitigation Strategies}

Based on comprehensive risk analysis, we propose three mitigation strategies:

\textbf{Strategy 1: Pre-Order Intelligence \& Adaptive Ordering}
\begin{itemize}
    \item Monitor 10-day weather forecasts and regulatory developments before finalizing order
    \item If adverse conditions detected, reduce $Q$ to range [245, 260] based on risk severity
    \item Default to conservative $Q \in [255, 260]$ to expand buffer and reduce loss probability
\end{itemize}

\textbf{Strategy 2: Supplier Relationship Management}
\begin{itemize}
    \item Negotiate higher refund rate $f$ with Leisure Limited (target 75\% vs current 50\%)
    \item Consider partial ordering: 200 units initially, option for 50 to 70 additional units as needed
\end{itemize}

\textbf{Strategy 3: Diversified Sales Channels}
\begin{itemize}
    \item Establish spot market relationships for post-holiday discount sales
    \item Explore regional fireworks retailers as secondary buyers for excess inventory
\end{itemize}

%==============================================================================
\subsection{Task 6: Corvette Prize Incentive}
%==============================================================================

Leisure Limited offers a \$40,000 Corvette prize to the stand with highest statewide sales. We use Excel-based conditional probability analysis to evaluate Q* under prize incentives.

\subsubsection{Ambiguity Resolution: Prize Rule Selection}
\begin{itemize}
    \item 5\% chance if sales $\geq 400$ units (primary)
    \item 3\% chance if sales $\geq 380$ units (other states context)
    \item 7\% chance if sales $\geq 420$ units (other states context)
\end{itemize}
We use the \textbf{5\% @ 400 units} rule as the baseline analysis, treating 380 and 420 thresholds as contextual references. This aligns with the emphasis on the 400-unit threshold.

\subsubsection{(a) Modified Profit Model}

Expected profit with prize incentive:
\[
\E[\Pi_{\text{total}}(Q)] = \E[\Pi_{\text{base}}(Q)] + \E[\text{Prize} \mid Q]
\]

where base profit follows standard newsvendor model:
\[
\E[\Pi_{\text{base}}(Q)] = p \cdot \E[\min(D,Q)] + f \cdot c \cdot \E[(Q-D)^+] - s \cdot \E[(Q-D)^+] - K - c \cdot Q
\]

Prize component depends on order quantity:
\[
\E[\text{Prize} \mid Q] = \begin{cases}
0 & \text{if } Q < 400 \\
\text{Prize} \times P(\text{win}) \times P(D \geq 400) & \text{if } Q \geq 400
\end{cases}
\]

For Uniform$(120, 420)$ demand:
\[
P(D \geq 400) = \frac{420 - 400}{420 - 120} = \frac{20}{300} = 0.0667
\]

\textbf{Expected prize at $Q \geq 400$:}
\[
\E[\text{Prize}] = \$40{,}000 \times 0.05 \times 0.0667 = \$133.33
\]

\subsubsection{(b) Optimal Quantity Q** Calculation}

\textbf{Conditional Probability Approach:}

For each candidate $Q$, we partition demand into regions and calculate conditional expected profits.

\textbf{Example: $Q = 400$}

Demand regions:
\begin{itemize}
    \item Region 1: $D < 400$ with probability $P(D < 400) = 280/300 = 0.9333$
    \item Region 2: $D \geq 400$ with probability $P(D \geq 400) = 20/300 = 0.0667$
\end{itemize}

\textbf{Region 1} ($D < 400$): Expected sales = $\frac{120 + 400}{2} = 260$
\[
\E[\Pi \mid D < 400] = 5(260) + 0.5(3)(140) - 0.5(140) - 20 - 3(400) = \$250.67
\]

\textbf{Region 2} ($D \geq 400$): Sales = 400, no leftover, prize eligible
\[
\E[\Pi \mid D \geq 400] = 5(400) - 20 - 3(400) + 40{,}000(0.05) = \$2{,}780
\]

\textbf{Total expected profit:}
\[
\E[\Pi_{\text{total}}(400)] = 0.9333(\$250.67) + 0.0667(\$2{,}780) = \$419.54
\]

\textbf{Candidate Evaluation:}

\begin{table}[H]
\centering
\small
\setlength{\tabcolsep}{5pt}
\begin{tabular}{ccccccc}
\toprule
\textbf{$Q$} & \textbf{$\E[\text{Sales}]$} & \textbf{$\E[\text{Leftover}]$} & \textbf{Base} & \textbf{Prize} & \textbf{Total} & \textbf{Note} \\
 & & & \textbf{Profit} & \textbf{EV} & \textbf{$\E[\Pi]$} & \\
\midrule
270 & 232.5 & 37.5 & \$370 & \$0 & \$370 & Baseline Q* \\
380 & 267.0 & 113.0 & \$289 & \$160 & \$449 & Above threshold \\
\rowcolor{lightgray} 400 & 269.0 & 131.0 & \$257 & \$214 & \textbf{\$471} & \textbf{Optimal Q**} \\
420 & 270.0 & 150.0 & \$220 & \$213 & \$433 & Max threshold \\
\bottomrule
\end{tabular}
\caption{Profit analysis at candidate order quantities (Excel-based calculations)}
\label{tab:q6_candidates}
\end{table}

\textbf{Optimal Decision:} $Q^{**} = 400$ units maximizes total expected profit at \$471.

\textit{Note:} Complete iteration table (Q = 120 to 420) available in the Excel file, confirming optimal at boundary.

\subsubsection{(c) Risk-Seeking Behavior Analysis}

\textbf{Comparison: $Q^*$ vs $Q^{**}$}

\begin{table}[H]
\centering
\small
\begin{tabular}{lccc}
\toprule
\textbf{Metric} & \textbf{$Q^* = 270$} & \textbf{$Q^{**} = 400$} & \textbf{Change} \\
\midrule
Order Quantity & 270 & 400 & +130 (+48\%) \\
Expected Sales & 232.5 & 269.0 & +36.5 \\
Expected Leftover & 37.5 & 131.0 & +93.5 (+249\%) \\
Base Profit & \$370 & \$257 & -\$113 \\
Expected Prize & \$0 & \$214 & +\$214 \\
\midrule
\textbf{Total E[Profit]} & \textbf{\$370} & \textbf{\$471} & \textbf{+\$101 (+27\%)} \\
\bottomrule
\end{tabular}
\caption{Prize incentive impact: Q* vs Q** (Q6)}
\label{tab:q6_comparison}
\end{table}
\begin{enumerate}
    \item \textbf{Significant inventory increase:} $Q^{**}$ is 48\% higher than baseline
    \item \textbf{Base profit deteriorates:} Aggressive overstocking reduces base profit by \$113 (-31\%)
    \item \textbf{Prize compensates:} Expected prize value (\$214) offsets base profit loss
    \item \textbf{Net benefit:} Total profit increases 27\% (\$370 $\to$ \$471)
\end{enumerate}
The prize incentive induces \textbf{strong risk-seeking behavior}:
\begin{itemize}
    \item SparkFire accepts 300\% increase in expected leftover inventory
    \item Base profitability declines, but prize eligibility compensates
    \item Decision shifts from conservative (balanced $C_u=C_o$) to aggressive for high-sales threshold.
\end{itemize}
\subsubsection{Behavioral Economics: EV Model Limitations}
Our model adds expected prize value (\$133 to \$267) to base profit, treating the Corvette as a monetary equivalent. This yields $Q^{**} = 420$.

\textbf{Behavioral Reality (Kahneman-Tversky Theory):}
Real decision-makers likely \textit{over-order beyond} $Q^{**} = 420$ due to:
\begin{enumerate}
    \item \textbf{Probability weight:} Small probabilities are psychologically overweighted: 5\% \textit{feels} 20\%
    \item \textbf{Framing effect:} Win a \$40,000 Corvette is vivid and appealing, abstract \$267 EV isn't
    \item \textbf{Regret aversion:} Fear of ``almost winning'' drives extra buffer ordering
    \item \textbf{Non-linear utility:} Marginal utility of \$40k windfall far exceeds utility of \$267 EV.
\end{enumerate}

\textbf{Practical Implication:}

Lottery-style incentives exploit behavioral biases. While $Q^{**} = 420$ is actuarially optimal, actual orders may reach 450 to 500 units as managers chase the psychologically compelling prize, sacrificing expected profit for emotional appeal.

%==============================================================================
\subsection{Task 7: Quantity Discounts}
%==============================================================================

The wholesaler offers all-units quantity discounts with tiered pricing:
\[
c(Q) = \begin{cases}
\$3.00 & \text{if } Q \in [1, 199] \\
\$2.85 & \text{if } Q \in [200, 399] \\
\$2.70 & \text{if } Q \geq 400
\end{cases}
\]

\subsubsection{Tier-by-Tier Analysis}

For each cost tier, we compute the unconstrained newsvendor optimal $Q^*$, then evaluate feasibility within tier bounds.

\begin{table}[H]
\centering
\small
\setlength{\tabcolsep}{5pt}
\begin{tabular}{lccccccc}
\toprule
\textbf{Tier} & \textbf{Unit} & \textbf{$C_o$} & \textbf{$C_u$} & \textbf{Critical} & \textbf{Unconstrained} & \textbf{In} & \textbf{Candidate} \\
\textbf{Range} & \textbf{Cost} & & & \textbf{Ratio} & \textbf{$Q^*$} & \textbf{Range?} & \textbf{$Q$} \\
\midrule
1--199 & \$3.00 & \$2.00 & \$2.00 & 0.5000 & 270.0 & No & 199 \\
200--399 & \$2.85 & \$1.93 & \$2.15 & 0.5276 & 278.3 & Yes & 278 \\
400+ & \$2.70 & \$1.85 & \$2.30 & 0.5542 & 286.3 & No & 400 \\
\bottomrule
\end{tabular}
\caption{Discount tier feasibility analysis (Q7)}
\label{tab:q7_tiers}
\end{table}

\begin{itemize}
    \item \textbf{Tier 1 (\$3.00):} Unconstrained $Q^* = 270$ exceeds tier maximum (199), so evaluate boundary $Q = 199$
    \item \textbf{Tier 2 (\$2.85):} Unconstrained $Q^* = 278$ falls within [200, 399], this is a feasible interior solution
    \item \textbf{Tier 3 (\$2.70):} Unconstrained $Q^* = 286$ below tier minimum (400), so evaluate boundary $Q = 400$
\end{itemize}

\subsubsection{Candidate Profit Comparison}
\begin{table}[H]
\centering
\small
\setlength{\tabcolsep}{6pt}
\begin{tabular}{cccccc}
\toprule
\textbf{$Q$} & \textbf{Unit Cost} & \textbf{$\E[\text{Sales}]$} & \textbf{$\E[\text{Leftover}]$} & \textbf{$\E[\text{Profit}]$} & \textbf{Note} \\
\midrule
199 & \$3.00 & 189 & 10 & \$336 & Tier 1 max \\
\rowcolor{lightgray} 278 & \$2.85 & 237 & 41 & \textbf{\$408} & Tier 2 optimal \\
400 & \$2.70 & 269 & 131 & \$358 & Tier 3 min \\
\bottomrule
\end{tabular}
\caption{Candidate order quantities and expected profits (Q7, Excel-based)}
\label{tab:q7_candidates}
\end{table}

\textbf{Optimal Decision:} $Q^*_d = 278$ units at \$2.85/unit
\begin{itemize}
    \item Middle tier (\$2.85) dominates despite not having the lowest unit cost
    \item Ordering 400 units to access \$2.70 pricing forces excessive overage (131 units expected leftover)
    \item Overage cost penalty outweighs \$0.15/unit savings: total cost increases by \$50.41
\end{itemize}

\subsubsection{Comparison to Baseline}

\begin{figure}[H]
\centering
\begin{minipage}{0.45\textwidth}
    \centering
    \small
    \begin{tabular}{rccc}
    \toprule
    \textbf{Scenario} & \textbf{$Q^*$} & \textbf{Unit Cost} & \textbf{$\E[\text{Profit}]$} \\
    \midrule
    Baseline & 270 & \$3.00 & \$370 \\
    Discount Tiers & 278 & \$2.85 & \$408 \\
    \midrule
    \textbf{Improvement} & +8 units & $-$\$0.15 & \textcolor{accentred}{+\$38} \\
    \bottomrule
    \end{tabular}
    \captionof{table}{Quantity discount benefit vs baseline (Q7)}
    \label{tab:q7_comparison}
\end{minipage}
\hfill
\begin{minipage}{0.43\textwidth}
    \centering
    \includegraphics[width=\textwidth]{../output/plots/q7_profit_by_tier.png}
    \caption{Expected profit across discount tiers}
    \label{fig:q7_profit}
\end{minipage}
\end{figure}

The quantity discount structure increases expected profit by 10\% while requiring minimal additional inventory (8 units).

\textbf{Supply Chain Coordination Insight:} Quantity discounts align supplier and buyer incentives by encouraging larger orders (beneficial for supplier's economies of scale) while sharing cost savings with the buyer. The tiered structure prevents extreme ordering behavior; the marginal benefit of the deepest discount (400+ tier) is insufficient to justify the inventory risk.

%==============================================================================
% End of Technical Appendix
%==============================================================================


%==============================================================================
\section{Technical Appendix}
%==============================================================================

\subsection*{Computational Methodology}

All analyses employ the newsvendor model framework with demand $D \sim \text{Uniform}(120, 420)$. Calculations are performed using:
\begin{itemize}
    \item \textbf{Python 3.11:} Analytical solutions, Monte Carlo simulations, and visualization (NumPy, Matplotlib)
    \item \textbf{Microsoft Excel:} Formula-based verification and sensitivity analysis
\end{itemize}

Results presented below are primarily from Python analytical solutions. Complete Excel workbook with detailed formula documentation is included in submission materials. Full CSV datasets available in \texttt{output/csv/} directory.

%==============================================================================
\subsection{Task 1: Conceptual Analysis of Order Quantity}
%==============================================================================

\subsubsection{Overage vs. Underage Trade-off Analysis}

With selling price $p = \$5$ and wholesale cost $c = \$3$:

\textbf{Cost of Underage (lost profit per stockout):}
\[
C_u = p - c = 5 - 3 = \$2.00
\]

Every unit of unmet demand costs \$2 in lost profit margin.

\textbf{Cost of Overage (net loss per unsold unit):}
\[
C_o = c(1-f) + s = 3(1-0.5) + 0.5 = \$2.00
\]

Every unsold unit costs \$2: we paid \$3, receive \$1.50 refund (50\% of cost), and pay \$0.50 shipping to return it.

\subsubsection{Ambiguity Resolution}

The overage cost $C_o$ represents the \textit{net loss} per unsold unit. While we receive a refund of $f \cdot c = \$1.50$, we incur a shipping cost of $s = \$0.50$ to return the unit. Therefore:
\begin{align*}
\text{Net loss} &= \text{Cost} - \text{Refund} + \text{Shipping} \\
&= c - fc + s = c(1-f) + s
\end{align*}

\subsubsection{Conceptual Prediction}

\textbf{Prediction:} The optimal order quantity $Q^*$ should \textbf{EQUAL} expected demand (270 units).

\textbf{Reasoning:}
\begin{itemize}
    \item $C_u = C_o = \$2.00$ creates perfectly balanced costs—understocking and overstocking are equally penalized
    \item This symmetric cost structure requires equal weighting of stockout and overage probabilities
    \item For uniform distribution, this balance occurs at the median, which equals the mean (270 units)
    \item Mathematically: we seek $P(D \geq Q) = P(D \leq Q) = 0.5$
\end{itemize}

This prediction will be verified analytically in Task 2.

%==============================================================================
\subsection{Task 2: Optimal Order Quantity Calculation}
%==============================================================================

\subsubsection{Newsvendor Model Formulation}

\textbf{Parameters:} $p = \$5$ (selling price), $c = \$3$ (unit cost), $f = 0.5$ (refund fraction), $s = \$0.50$ (shipping cost per return), $K = \$20$ (fixed ordering cost).

\textbf{Cost Structure:}
\begin{align*}
C_u &= p - c = 5 - 3 = \$2.00 \quad \text{(underage cost: lost profit per stockout)}\\
C_o &= c(1-f) + s = 3(1-0.5) + 0.5 = \$2.00 \quad \text{(overage cost: net loss per unsold unit)}
\end{align*}

\textbf{Critical Ratio:}
\[
\text{CR} = \frac{C_u}{C_u + C_o} = \frac{2.00}{2.00 + 2.00} = 0.5000
\]

\textbf{Optimal Order Quantity:}
\[
Q^* = a + (b-a) \times \text{CR} = 120 + (420-120) \times 0.5 = \boxed{270 \text{ units}}
\]

\textbf{Verification:} This confirms our conceptual prediction from Task 1. The balanced cost structure ($C_u = C_o$) results in $Q^*$ exactly at the expected demand.

\subsubsection{Profit Breakdown at $Q^* = 270$}

\begin{figure}[H]
\centering
\begin{minipage}{0.48\textwidth}
    \centering
    \small
    \setlength{\tabcolsep}{4pt}
    \begin{tabular}{lrl}
    \toprule
    \textbf{Component} & \textbf{Value} & \textbf{Calculation} \\
    \midrule
    Expected Sales & 232.50 units & $\E[\min(D, Q^*)]$ \\
    Expected Leftover & 37.50 units & $Q^* - \E[\text{Sales}]$ \\
    \midrule
    Revenue & \$1,162.50 & $232.50 \times \$5$ \\
    Salvage Value & \$56.25 & $37.50 \times \$3 \times 0.5$ \\
    Shipping Cost & $-$\$18.75 & $37.50 \times \$0.50$ \\
    Ordering Cost & $-$\$20.00 & Fixed \\
    Variable Cost & $-$\$810.00 & $270 \times \$3$ \\
    \midrule
    \textcolor{accentred}{\textbf{Expected Profit}} & \textcolor{accentred}{\textbf{\$370.00}} & Total \\
    \bottomrule
    \end{tabular}
    \captionof{table}{Profit decomposition at $Q^* = 270$ units}
\end{minipage}
\hfill
\begin{minipage}{0.45\textwidth}
    \centering
    \includegraphics[width=\textwidth]{../output/plots/q1_q2_profit_curve.png}
    \caption{Expected profit curve with $Q^*$}
\end{minipage}
\end{figure}

\subsubsection{Distribution Sensitivity Analysis}

The optimal solution depends on the assumed demand distribution. Below we compare $Q^*$ and expected profit across alternative distributions with similar central tendency:

\begin{table}[H]
\centering
\small
\begin{tabular}{lccc}
\toprule
\textbf{Distribution} & \textbf{Parameters} & \textbf{$Q^*$} & \textbf{$\E[\text{Profit}]$} \\
\midrule
Uniform & $[120, 420]$ & 270 & \$370 \\
Normal & $\mu=270, \sigma=50$ & 270 & \$385 \\
Lognormal & $\mu=270$, right-skewed & 285 & \$368 \\
Triangular & $[120, 270, 420]$ & 255 & \$372 \\
\bottomrule
\end{tabular}
\caption{Impact of distribution choice on optimal policy (same cost parameters)}
\label{tab:dist_sensitivity}
\end{table}
Normal distribution yields higher profit due to concentrated probability around the mean. Lognormal (right-skewed) shifts $Q^*$ upward to hedge against high-demand tail risk. Triangular (mode-centered) reduces optimal order slightly. Distribution choice materially affects both policy and performance.

%==============================================================================
\subsection{Task 3: Refund Sensitivity Analysis}
%==============================================================================

We evaluate how refund generosity affects optimal ordering decisions across the full spectrum from no refunds to full refunds: $f \in \{0.00, 0.25, 0.50, 0.75, 1.00\}$.

\subsubsection{Sensitivity Results}

\begin{table}[H]
\centering
\small
\setlength{\tabcolsep}{4pt}
\begin{tabular}{ccccccc}
\toprule
\textbf{Refund} & \textbf{$C_o$} & \textbf{$C_u$} & \textbf{Critical} & \textbf{$Q^*$} & \textbf{$\E[\text{Profit}]$} \\
\textbf{Rate $f$} & & & \textbf{Ratio} & \textbf{(units)} & \\
\midrule
0.00 & \$3.50 & \$2.00 & 0.3636 & 229.1 & \$329.09 \\
0.25 & \$2.75 & \$2.00 & 0.4211 & 246.3 & \$346.32 \\
0.50 & \$2.00 & \$2.00 & 0.5000 & 270.0 & \$370.00 \\
0.75 & \$1.25 & \$2.00 & 0.6154 & 304.6 & \$404.62 \\
1.00 & \$0.50 & \$2.00 & 0.8000 & 360.0 & \$460.00 \\
\bottomrule
\end{tabular}
\caption{Refund sensitivity analysis across full spectrum}
\label{tab:q3_sensitivity}
\end{table}

\textbf{Observations:}
\begin{itemize}
    \item Higher refund rates systematically reduce overage cost $C_o$, increasing critical ratio and $Q^*$
    \item Boundary cases show full range: $Q^*$ from 229 units (no refund) to 360 units (full refund)
    \item Expected profit increases monotonically from \$329 ($f=0$) to \$460 ($f=1$)—a \$131 gain
    \item Full refund ($f=1.00$) essentially eliminates overage risk ($C_o = \$0.50$ shipping only), encouraging aggressive ordering
\end{itemize}

\begin{figure}[H]
\centering
\begin{minipage}{0.48\textwidth}
    \centering
    \includegraphics[width=\textwidth]{../output/plots/q3_Qstar_vs_refund.png}
    \caption*{(a) Optimal $Q^*$ vs refund rate}
\end{minipage}
\hfill
\begin{minipage}{0.48\textwidth}
    \centering
    \includegraphics[width=\textwidth]{../output/plots/q3_profit_vs_refund.png}
    \caption*{(b) Expected profit vs refund rate}
\end{minipage}
\caption{Impact of refund policy on order quantity and profitability}
\label{fig:q3_sensitivity}
\end{figure}

%==============================================================================
\subsection{Task 4: Pricing Decision}
%==============================================================================

We compare two pricing strategies : $p = \$5$ (baseline) vs $p = \$6$ (premium pricing).

\subsubsection{Conceptual Prediction for $p = \$6$}

\textbf{At $p = \$6$:}
\begin{itemize}
    \item $C_u = p - c = 6 - 3 = \$3$ (increased from \$2)
    \item $C_o = c(1-f) + s = \$2$ (unchanged)
    \item Critical ratio: $C_u/(C_u + C_o) = 3/5 = 0.60 > 0.50$
\end{itemize}

\textbf{Prediction:} $Q^*$ should exceed 270 units. Higher underage cost shifts strategy toward more inventory to reduce stockout risk.

\subsubsection{Comparative Analysis}

\begin{table}[H]
\centering
\small
\setlength{\tabcolsep}{6pt}
\begin{tabular}{lcccccc}
\toprule
\textbf{Price} & \textbf{$C_u$} & \textbf{Critical} & \textbf{$Q^*$} & \textbf{$\E[\text{Sales}]$} & \textbf{$\E[\text{Leftover}]$} & \textbf{$\E[\text{Profit}]$} \\
& & \textbf{Ratio} & \textbf{(units)} & \textbf{(units)} & \textbf{(units)} & \\
\midrule
\$5 (baseline) & \$2.00 & 0.5000 & 270.0 & 232.50 & 37.50 & \$370.00 \\
\$6 & \$3.00 & 0.6000 & 300.0 & 246.00 & 54.00 & \$610.00 \\
\midrule
\multicolumn{6}{r}{\textbf{Profit Increase:}} & \textcolor{accentred}{+\$240.00 (+64.9\%)} \\
\bottomrule
\end{tabular}
\caption{Pricing comparison: \$5 vs \$6 (Q4)}
\label{tab:q4_pricing}
\end{table}

\textbf{Recommendation:} Set $p = \$6$. The higher price increases expected profit by 65\% while requiring only 11\% more inventory (300 vs 270 units). The increased underage cost ($C_u$ rises from \$2 to \$3) justifies stocking more units to capture higher per-unit margins.

\subsubsection{Price Elasticity Sensitivity Analysis}

\textbf{Critical Assumption:} The preceding analysis assumes demand is \textbf{price-inelastic} (demand remains Uniform$(120, 420)$ regardless of price). This is unrealistic for most products.

\textbf{Real-World Consideration:} A price increase from \$5 to \$6 (20\% hike) would likely reduce demand. Let's explore three plausible elasticity scenarios:

\begin{table}[H]
\centering
\small
\setlength{\tabcolsep}{5pt}
\begin{tabular}{llcccc}
\toprule
\textbf{Scenario} & \textbf{Demand} & \textbf{Mean} & \textbf{$Q^*$} & \textbf{$\E[\Pi]$} & \textbf{vs \$5} \\
& \textbf{Distribution} & \textbf{Demand} & & \textbf{@ \$6} & \textbf{baseline} \\
\midrule
\textit{Baseline (p=\$5)} & Uniform(120, 420) & 270 & 270 & \$370 & --- \\[0.5ex]
\midrule
A: Inelastic & Uniform(120, 420) & 270 & 300 & \$610 & \textcolor{primaryblue}{+65\%} \\
B: Moderate & Uniform(96, 336) & 216 & 240 & \$448 & \textcolor{primaryblue}{+21\%} \\
C: High elasticity & Uniform(72, 252) & 162 & 180 & \$286 & \textcolor{accentred}{$-$23\%} \\
\bottomrule
\end{tabular}
\caption{Price elasticity scenarios at $p = \$6$ (Q4 sensitivity)}
\label{tab:q4_elasticity}
\end{table}

\textbf{Scenario Details:}
\begin{itemize}
    \item \textbf{A (Inelastic):} Demand unchanged—upper bound on \$6 profit
    \item \textbf{B (Moderate):} 20\% demand reduction—mean drops to 216
    \item \textbf{C (High elasticity):} 40\% demand reduction—common for discretionary products
\end{itemize}

\textbf{Strategic Implications:}
\begin{enumerate}
    \item \textbf{Moderate elasticity} (Scenario B) still favors \$6 pricing with +21\% profit gain
    \item \textbf{High elasticity} (Scenario C) makes \$6 pricing detrimental—profit falls 23\% below \$5 baseline
    \item \textbf{Decision criterion:} Price elasticity of demand must be better than $-$2.0 for \$6 to outperform \$5
\end{enumerate}

\textbf{Recommendation:} Before implementing \$6 pricing, conduct market research or A/B testing to estimate true price elasticity. If elasticity is moderate ($|\varepsilon| < 1.0$), proceed with premium pricing. If highly elastic ($|\varepsilon| > 2.0$), maintain \$5 pricing to preserve volume.

%==============================================================================
\subsection{Task 5: Risk \& Simulation Analysis}
%==============================================================================
\subsubsection{Ambiguity Resolution - Continuous vs Discrete Demand}

For simulation realism, we use discrete uniform demand $D \in \{120, 121, \ldots, 420\}$ with equal probability (1/301 each). Analytical tasks (1--4, 6--8) use continuous approximation.

\subsubsection{Simulation Setup and Random Number Generation}

To simulate demand realizations, we use Python's \texttt{random.randint(a, b)} function, which generates discrete uniform random integers over $[a, b]$ with equal probability $1/(b-a+1)$

\textbf{Algorithm:}
\begin{enumerate}
    \item \textbf{Seed:} Set \texttt{random.seed(6334)} to ensure reproducibility and identical sequence of random numbers
    \item \textbf{Demand generation:} For each of 500 trials, generate $D_i \in \{120, 121, \ldots, 420\}$
    \item \textbf{Profit calculation:} Compute $\Pi(Q^*, D_i)$ using:
    \begin{itemize}
        \item Sales: $\min(D_i, Q^*)$
        \item Leftover: $\max(0, Q^* - D_i)$
        \item Profit: $p \cdot \text{Sales} + f \cdot c \cdot \text{Leftover} - s \cdot \text{Leftover} - K - c \cdot Q^*$
    \end{itemize}
\end{enumerate}

\textbf{Note:} Excel workbook provides iteration-level detail.

\subsubsection{Multi-Seed Robustness Verification}

We verify the simulation's robustness by running with three different seeds:

\begin{table}[H]
\centering
\small
\setlength{\tabcolsep}{4pt}
\begin{tabular}{cccccc}
\toprule
\textbf{Random} & \textbf{Mean} & \textbf{Std Dev} & \textbf{Min} & \textbf{P(Loss)} & \textbf{5th Pct} \\
\textbf{Seed} & \textbf{Profit} & & \textbf{Profit} & & \\
\midrule
6334 & \$373.00 & \$194.45 & -\$80.00 & 6.8\% & \$-22.60 \\
1234 & \$379.12 & \$191.72 & -\$80.00 & 5.4\% & \$-7.80 \\
5678 & \$387.86 & \$189.09 & -\$76.00 & 6.2\% & \$5.20 \\
\midrule
\textbf{Average} & \textbf{\$380.00} & \textbf{\$191.75} & & \textbf{6.1\%} & \\
\bottomrule
\end{tabular}
\caption{Multi-seed robustness check (Q5)}
\label{tab:q5_multiseed}
\end{table}

\textbf{Conclusion:} Mean profits cluster tightly around theoretical \$370, confirming simulation validity. The 6.1\% average loss probability indicates moderate downside risk under baseline assumptions.

\begin{figure}[H]
\centering
\includegraphics[width=0.9\textwidth]{../output/plots/q5_multiseed_comparison.png}
\caption{Profit distribution across three random seeds (Q5)}
\label{fig:q5_multiseed}
\end{figure}

\subsubsection{Excel Implementation: Single-Iteration Transparency}

While Python provides aggregate statistics across 500 trials, the Excel workbook offers iteration-level visibility for pedagogical transparency and formula verification.

\begin{figure}[H]
\centering
\begin{minipage}{0.48\textwidth}
    \centering
    \includegraphics[width=\textwidth]{images/q5_excel_first5iter.png}
    \caption*{(a) First 5 of 500 iterations}
\end{minipage}
\hfill
\begin{minipage}{0.48\textwidth}
    \centering
    \includegraphics[width=\textwidth]{images/q5_excel_sim_results.png}
    \caption*{(b) Summary statistics (500 trials)}
\end{minipage}

\vspace{0.5em}

\begin{minipage}{0.48\textwidth}
    \centering
    \includegraphics[width=\textwidth]{images/q5_expvsactualprofit.png}
    \caption*{(c) Expected vs actual profit}
\end{minipage}
\hfill
\begin{minipage}{0.48\textwidth}
    \centering
    \includegraphics[width=\textwidth]{images/q5_netprofit_500sims_excel.png}
    \caption*{(d) Net profit distribution}
\end{minipage}
\caption{Excel simulation detail: iteration-level transparency (Q5)}
\label{fig:q5_excel}
\end{figure}

Excel results confirm Python findings: mean profit \$370--\$380 range, 6--7\% loss probability, substantial profit variance (\$190+ std dev).

\subsubsection{Break-Even Analysis and Conservative Ordering Strategy}

\textbf{Critical Question:} At what demand does profit become zero?

Setting $\Pi(Q, D) = 0$ and solving for $D$:
\begin{align*}
p \cdot D + f \cdot c \cdot (Q - D) - s \cdot (Q - D) - K - c \cdot Q &= 0 \\
D \cdot (p - fc + s) &= K + Q \cdot (c - fc + s) \\
D^* &= \frac{K + Q \cdot (c - fc + s)}{p - fc + s}
\end{align*}

For $Q^* = 270$: $D^* = \frac{20 + 270(3 - 1.5 + 0.5)}{5 - 1.5 + 0.5} = \frac{560}{4} = \boxed{140}$ cases.

Profit becomes zero at $D = 140$. With minimum demand at 120, only a \textbf{20-unit buffer} exists—just 6.7\% of the 300-unit demand range. This narrow margin makes the optimal policy vulnerable to even slight demand underperformance.

\textbf{Conservative Strategy Evaluation:}

To expand the buffer zone, we evaluate order quantities below $Q^* = 270$. Lower $Q$ raises the break-even demand point, creating more cushion above the minimum. 

\begin{table}[H]
\centering
\small
\setlength{\tabcolsep}{3pt}
\begin{tabular}{ccccccc}
\toprule
\textbf{$Q$} & \textbf{Break-even $D$} & \textbf{Buffer} & \textbf{Mean Profit} & \textbf{Std Dev} & \textbf{P(Loss)} & \textbf{Min Profit} \\
\midrule
255 & 131 & 11 units (3.7\%) & \$370.42 & \$171.24 & 4.2\% & -\$50.00 \\
260 & 135 & 15 units (5.0\%) & \$371.17 & \$175.42 & 5.0\% & -\$60.00 \\
265 & 138 & 18 units (6.0\%) & \$371.91 & \$179.51 & 5.8\% & -\$70.00 \\
\rowcolor{lightgray} 270 & 140 & 20 units (6.7\%) & \textbf{\$372.64} & \$183.53 & 6.8\% & -\$80.00 \\
\bottomrule
\end{tabular}
\caption{Conservative strategy evaluation: buffer vs profit trade-off (Q5)}
\label{tab:q5_conservative}
\end{table}

\textbf{Analysis:}
\begin{itemize}
    \item Reducing $Q$ from 270 to 255 expands buffer from 6.7\% to 3.7\% of demand range
    \item Profit sacrifice minimal: \$372.64 → \$370.42 (only \$2.22 or 0.6\%)
    \item P(Loss) drops from 6.8\% to 4.2\% (38\% relative reduction)
    \item Worst-case improves by \$30 (-\$50 vs -\$80)
\end{itemize}

\textbf{Recommendation:} 
Order $Q \in [255, 260]$ to balance risk mitigation with profit preservation. The \$2 to \$3 profit sacrifice buys substantial downside protection, appropriate for risk-averse or capital-constrained operations.

\subsubsection{Demand Shock Scenarios: Weather and Regulatory Risk}

All prior analysis assumes demand remains Uniform$(120, 420)$ regardless of external conditions. This is unrealistic.
\begin{itemize}
    \item \textbf{Weather:} Heavy rain during July 4th weekend reduces outdoor celebrations
    \item \textbf{Regulatory:} Sudden fireworks ban due to drought/fire risk
\end{itemize}

If demand drops while $Q^* = 270$ is already ordered, SparkFire faces overstocking losses.

\begin{table}[H]
\centering
\small
\setlength{\tabcolsep}{4pt}
\begin{tabular}{lcccccc}
\toprule
\textbf{Scenario} & \textbf{Demand} & \textbf{Mean} & \textbf{Mean} & \textbf{Profit} & \textbf{Min} & \textbf{P(Loss)} \\
 & \textbf{Reduction} & \textbf{Demand} & \textbf{Profit} & \textbf{Change} & \textbf{Profit} & \\
\midrule
Baseline & 0\% & 270 & \$373.00 & --- & -\$80 & 6.8\% \\
Mild shock & 10\% & 243 & \$328.91 & -11.8\% & -\$128 & 12.0\% \\
Moderate (weather) & 20\% & 216 & \$259.95 & \textcolor{accentred}{-30.3\%} & -\$176 & 18.6\% \\
Severe & 40\% & 162 & \$95.82 & \textcolor{accentred}{-74.3\%} & -\$272 & 36.8\% \\
Catastrophic (ban) & 60\% & 108 & \textcolor{accentred}{-\$132.28} & \textcolor{accentred}{-135.5\%} & -\$368 & \textcolor{accentred}{75.6\%} \\
\bottomrule
\end{tabular}
\caption{Demand shock impact on profitability with fixed Q* = 270}
\label{tab:q5_shock}
\end{table}

\textbf{Critical Findings:}
\begin{itemize}
    \item 20\% demand reduction (weather) leads to 30\% profit erosion to \$216
    \item 60\% demand reduction (regulatory) has expected \textbf{loss} of \$132
    \item The profit function is \textbf{highly sensitive} to demand shocks when $Q$ is fixed
\end{itemize}

\subsubsection{Risk Mitigation Strategies}

Based on comprehensive risk analysis, we propose three mitigation strategies:

\textbf{Strategy 1: Pre-Order Intelligence \& Adaptive Ordering}
\begin{itemize}
    \item Monitor 10-day weather forecasts and regulatory developments before finalizing order
    \item If adverse conditions detected, reduce $Q$ to range [245, 260] based on risk severity
    \item Default to conservative $Q \in [255, 260]$ to expand buffer and reduce loss probability
\end{itemize}

\textbf{Strategy 2: Supplier Relationship Management}
\begin{itemize}
    \item Negotiate higher refund rate $f$ with Leisure Limited (target 75\% vs current 50\%)
    \item Consider partial ordering: 200 units initially, option for 50 to 70 additional units as needed
\end{itemize}

\textbf{Strategy 3: Diversified Sales Channels}
\begin{itemize}
    \item Establish spot market relationships for post-holiday discount sales
    \item Explore regional fireworks retailers as secondary buyers for excess inventory
\end{itemize}

%==============================================================================
\subsection{Task 6: Corvette Prize Incentive}
%==============================================================================

Leisure Limited offers a \$40,000 Corvette prize to the stand with highest statewide sales. We use Excel-based conditional probability analysis to evaluate Q* under prize incentives.

\subsubsection{Ambiguity Resolution: Prize Rule Selection}
\begin{itemize}
    \item 5\% chance if sales $\geq 400$ units (primary)
    \item 3\% chance if sales $\geq 380$ units (other states context)
    \item 7\% chance if sales $\geq 420$ units (other states context)
\end{itemize}
We use the \textbf{5\% @ 400 units} rule as the baseline analysis, treating 380 and 420 thresholds as contextual references. This aligns with the emphasis on the 400-unit threshold.

\subsubsection{(a) Modified Profit Model}

Expected profit with prize incentive:
\[
\E[\Pi_{\text{total}}(Q)] = \E[\Pi_{\text{base}}(Q)] + \E[\text{Prize} \mid Q]
\]

where base profit follows standard newsvendor model:
\[
\E[\Pi_{\text{base}}(Q)] = p \cdot \E[\min(D,Q)] + f \cdot c \cdot \E[(Q-D)^+] - s \cdot \E[(Q-D)^+] - K - c \cdot Q
\]

Prize component depends on order quantity:
\[
\E[\text{Prize} \mid Q] = \begin{cases}
0 & \text{if } Q < 400 \\
\text{Prize} \times P(\text{win}) \times P(D \geq 400) & \text{if } Q \geq 400
\end{cases}
\]

For Uniform$(120, 420)$ demand:
\[
P(D \geq 400) = \frac{420 - 400}{420 - 120} = \frac{20}{300} = 0.0667
\]

\textbf{Expected prize at $Q \geq 400$:}
\[
\E[\text{Prize}] = \$40{,}000 \times 0.05 \times 0.0667 = \$133.33
\]

\subsubsection{(b) Optimal Quantity Q** Calculation}

\textbf{Conditional Probability Approach:}

For each candidate $Q$, we partition demand into regions and calculate conditional expected profits.

\textbf{Example: $Q = 400$}

Demand regions:
\begin{itemize}
    \item Region 1: $D < 400$ with probability $P(D < 400) = 280/300 = 0.9333$
    \item Region 2: $D \geq 400$ with probability $P(D \geq 400) = 20/300 = 0.0667$
\end{itemize}

\textbf{Region 1} ($D < 400$): Expected sales = $\frac{120 + 400}{2} = 260$
\[
\E[\Pi \mid D < 400] = 5(260) + 0.5(3)(140) - 0.5(140) - 20 - 3(400) = \$250.67
\]

\textbf{Region 2} ($D \geq 400$): Sales = 400, no leftover, prize eligible
\[
\E[\Pi \mid D \geq 400] = 5(400) - 20 - 3(400) + 40{,}000(0.05) = \$2{,}780
\]

\textbf{Total expected profit:}
\[
\E[\Pi_{\text{total}}(400)] = 0.9333(\$250.67) + 0.0667(\$2{,}780) = \$419.54
\]

\textbf{Candidate Evaluation:}

\begin{table}[H]
\centering
\small
\setlength{\tabcolsep}{5pt}
\begin{tabular}{ccccccc}
\toprule
\textbf{$Q$} & \textbf{$\E[\text{Sales}]$} & \textbf{$\E[\text{Leftover}]$} & \textbf{Base} & \textbf{Prize} & \textbf{Total} & \textbf{Note} \\
 & & & \textbf{Profit} & \textbf{EV} & \textbf{$\E[\Pi]$} & \\
\midrule
270 & 232.5 & 37.5 & \$370 & \$0 & \$370 & Baseline Q* \\
380 & 267.0 & 113.0 & \$289 & \$160 & \$449 & Above threshold \\
\rowcolor{lightgray} 400 & 269.0 & 131.0 & \$257 & \$214 & \textbf{\$471} & \textbf{Optimal Q**} \\
420 & 270.0 & 150.0 & \$220 & \$213 & \$433 & Max threshold \\
\bottomrule
\end{tabular}
\caption{Profit analysis at candidate order quantities (Excel-based calculations)}
\label{tab:q6_candidates}
\end{table}

\textbf{Optimal Decision:} $Q^{**} = 400$ units maximizes total expected profit at \$471.

\textit{Note:} Complete iteration table (Q = 120 to 420) available in the Excel file, confirming optimal at boundary.

\subsubsection{(c) Risk-Seeking Behavior Analysis}

\textbf{Comparison: $Q^*$ vs $Q^{**}$}

\begin{table}[H]
\centering
\small
\begin{tabular}{lccc}
\toprule
\textbf{Metric} & \textbf{$Q^* = 270$} & \textbf{$Q^{**} = 400$} & \textbf{Change} \\
\midrule
Order Quantity & 270 & 400 & +130 (+48\%) \\
Expected Sales & 232.5 & 269.0 & +36.5 \\
Expected Leftover & 37.5 & 131.0 & +93.5 (+249\%) \\
Base Profit & \$370 & \$257 & -\$113 \\
Expected Prize & \$0 & \$214 & +\$214 \\
\midrule
\textbf{Total E[Profit]} & \textbf{\$370} & \textbf{\$471} & \textbf{+\$101 (+27\%)} \\
\bottomrule
\end{tabular}
\caption{Prize incentive impact: Q* vs Q** (Q6)}
\label{tab:q6_comparison}
\end{table}
\begin{enumerate}
    \item \textbf{Significant inventory increase:} $Q^{**}$ is 48\% higher than baseline
    \item \textbf{Base profit deteriorates:} Aggressive overstocking reduces base profit by \$113 (-31\%)
    \item \textbf{Prize compensates:} Expected prize value (\$214) offsets base profit loss
    \item \textbf{Net benefit:} Total profit increases 27\% (\$370 $\to$ \$471)
\end{enumerate}
The prize incentive induces \textbf{strong risk-seeking behavior}:
\begin{itemize}
    \item SparkFire accepts 300\% increase in expected leftover inventory
    \item Base profitability declines, but prize eligibility compensates
    \item Decision shifts from conservative (balanced $C_u=C_o$) to aggressive for high-sales threshold.
\end{itemize}
\subsubsection{Behavioral Economics: EV Model Limitations}
Our model adds expected prize value (\$133 to \$267) to base profit, treating the Corvette as a monetary equivalent. This yields $Q^{**} = 420$.

\textbf{Behavioral Reality (Kahneman-Tversky Theory):}
Real decision-makers likely \textit{over-order beyond} $Q^{**} = 420$ due to:
\begin{enumerate}
    \item \textbf{Probability weight:} Small probabilities are psychologically overweighted: 5\% \textit{feels} 20\%
    \item \textbf{Framing effect:} Win a \$40,000 Corvette is vivid and appealing, abstract \$267 EV isn't
    \item \textbf{Regret aversion:} Fear of ``almost winning'' drives extra buffer ordering
    \item \textbf{Non-linear utility:} Marginal utility of \$40k windfall far exceeds utility of \$267 EV.
\end{enumerate}

\textbf{Practical Implication:}

Lottery-style incentives exploit behavioral biases. While $Q^{**} = 420$ is actuarially optimal, actual orders may reach 450 to 500 units as managers chase the psychologically compelling prize, sacrificing expected profit for emotional appeal.

%==============================================================================
\subsection{Task 7: Quantity Discounts}
%==============================================================================

The wholesaler offers all-units quantity discounts with tiered pricing:
\[
c(Q) = \begin{cases}
\$3.00 & \text{if } Q \in [1, 199] \\
\$2.85 & \text{if } Q \in [200, 399] \\
\$2.70 & \text{if } Q \geq 400
\end{cases}
\]

\subsubsection{Tier-by-Tier Analysis}

For each cost tier, we compute the unconstrained newsvendor optimal $Q^*$, then evaluate feasibility within tier bounds.

\begin{table}[H]
\centering
\small
\setlength{\tabcolsep}{5pt}
\begin{tabular}{lccccccc}
\toprule
\textbf{Tier} & \textbf{Unit} & \textbf{$C_o$} & \textbf{$C_u$} & \textbf{Critical} & \textbf{Unconstrained} & \textbf{In} & \textbf{Candidate} \\
\textbf{Range} & \textbf{Cost} & & & \textbf{Ratio} & \textbf{$Q^*$} & \textbf{Range?} & \textbf{$Q$} \\
\midrule
1--199 & \$3.00 & \$2.00 & \$2.00 & 0.5000 & 270.0 & No & 199 \\
200--399 & \$2.85 & \$1.93 & \$2.15 & 0.5276 & 278.3 & Yes & 278 \\
400+ & \$2.70 & \$1.85 & \$2.30 & 0.5542 & 286.3 & No & 400 \\
\bottomrule
\end{tabular}
\caption{Discount tier feasibility analysis (Q7)}
\label{tab:q7_tiers}
\end{table}

\begin{itemize}
    \item \textbf{Tier 1 (\$3.00):} Unconstrained $Q^* = 270$ exceeds tier maximum (199), so evaluate boundary $Q = 199$
    \item \textbf{Tier 2 (\$2.85):} Unconstrained $Q^* = 278$ falls within [200, 399], this is a feasible interior solution
    \item \textbf{Tier 3 (\$2.70):} Unconstrained $Q^* = 286$ below tier minimum (400), so evaluate boundary $Q = 400$
\end{itemize}

\subsubsection{Candidate Profit Comparison}
\begin{table}[H]
\centering
\small
\setlength{\tabcolsep}{6pt}
\begin{tabular}{cccccc}
\toprule
\textbf{$Q$} & \textbf{Unit Cost} & \textbf{$\E[\text{Sales}]$} & \textbf{$\E[\text{Leftover}]$} & \textbf{$\E[\text{Profit}]$} & \textbf{Note} \\
\midrule
199 & \$3.00 & 189 & 10 & \$336 & Tier 1 max \\
\rowcolor{lightgray} 278 & \$2.85 & 237 & 41 & \textbf{\$408} & Tier 2 optimal \\
400 & \$2.70 & 269 & 131 & \$358 & Tier 3 min \\
\bottomrule
\end{tabular}
\caption{Candidate order quantities and expected profits (Q7, Excel-based)}
\label{tab:q7_candidates}
\end{table}

\textbf{Optimal Decision:} $Q^*_d = 278$ units at \$2.85/unit
\begin{itemize}
    \item Middle tier (\$2.85) dominates despite not having the lowest unit cost
    \item Ordering 400 units to access \$2.70 pricing forces excessive overage (131 units expected leftover)
    \item Overage cost penalty outweighs \$0.15/unit savings: total cost increases by \$50.41
\end{itemize}

\subsubsection{Comparison to Baseline}

\begin{figure}[H]
\centering
\begin{minipage}{0.45\textwidth}
    \centering
    \small
    \begin{tabular}{rccc}
    \toprule
    \textbf{Scenario} & \textbf{$Q^*$} & \textbf{Unit Cost} & \textbf{$\E[\text{Profit}]$} \\
    \midrule
    Baseline & 270 & \$3.00 & \$370 \\
    Discount Tiers & 278 & \$2.85 & \$408 \\
    \midrule
    \textbf{Improvement} & +8 units & $-$\$0.15 & \textcolor{accentred}{+\$38} \\
    \bottomrule
    \end{tabular}
    \captionof{table}{Quantity discount benefit vs baseline (Q7)}
    \label{tab:q7_comparison}
\end{minipage}
\hfill
\begin{minipage}{0.43\textwidth}
    \centering
    \includegraphics[width=\textwidth]{../output/plots/q7_profit_by_tier.png}
    \caption{Expected profit across discount tiers}
    \label{fig:q7_profit}
\end{minipage}
\end{figure}

The quantity discount structure increases expected profit by 10\% while requiring minimal additional inventory (8 units).

\textbf{Supply Chain Coordination Insight:} Quantity discounts align supplier and buyer incentives by encouraging larger orders (beneficial for supplier's economies of scale) while sharing cost savings with the buyer. The tiered structure prevents extreme ordering behavior; the marginal benefit of the deepest discount (400+ tier) is insufficient to justify the inventory risk.

%==============================================================================
% End of Technical Appendix
%==============================================================================


%==============================================================================
% Future sections placeholder
%==============================================================================

% \section{Python Code}
% \section{Reproducibility Notes}

\end{document}
